%-----------------------------------------------------------------------------
%
%               Template for sigplanconf LaTeX Class
%
% Name:         sigplanconf-template.tex
%
% Purpose:      A template for sigplanconf.cls, which is a LaTeX 2e class
%               file for SIGPLAN conference proceedings.
%
% Guide:        Refer to "Author's Guide to the ACM SIGPLAN Class,"
%               sigplanconf-guide.pdf
%
% Author:       Paul C. Anagnostopoulos
%               Windfall Software
%               978 371-2316
%               paul@windfall.com
%
% Created:      15 February 2005
%
%-----------------------------------------------------------------------------


\documentclass[preprint]{sigplanconf}

% The following \documentclass options may be useful:

% preprint      Remove this option only once the paper is in final form.
% 10pt          To set in 10-point type instead of 9-point.
% 11pt          To set in 11-point type instead of 9-point.
% authoryear    To obtain author/year citation style instead of numeric.

\usepackage{amsmath}
\usepackage{natbib}
\usepackage{amsmath}
\usepackage{amssymb}
\usepackage{amsthm}
\usepackage[utf8]{inputenc}
\usepackage{parskip}
\usepackage{color}
\usepackage{url}
\usepackage{enumerate}  % Needed by Pandoc

\theoremstyle{definition}
\newtheorem{theorem}{Theorem}
\newtheorem{lemma}{Lemma}
\newtheorem{definition}{Definition}

\newif\ifbeamer
%\beamertrue % Uncomment when using Beamer

\newif\iflsthaskell
\lsthaskelltrue % Uncomment when using listings and Haskell

%%%%%%%%%%%%%%%%%%%%%%%%%%%%%%%%%%%%%%%%%%%%%%%%%%%%%%%%%%%%%%%%%%%%%%%%%%%%%%%%%%%%%%%%%%%%%%%%%%%%

\ifbeamer

  \setbeamertemplate{blocks}[rounded][shadow=true]
  \setbeamercolor{block body}{bg=black!17}
  \hypersetup{colorlinks,linkcolor=blue,anchorcolor=blue,citecolor=blue,filecolor=blue,urlcolor=blue}

\else

  %\usepackage[colorlinks,linkcolor=blue,anchorcolor=blue,citecolor=blue,filecolor=blue,linkcolor=blue,urlcolor=blue]{hyperref}
  \urlstyle{sf}

\fi

%%%%%%%%%%%%%%%%%%%%%%%%%%%%%%%%%%%%%%%%%%%%%%%%%%%%%%%%%%%%%%%%%%%%%%%%%%%%%%%%%%%%%%%%%%%%%%%%%%%%

\definecolor{haskell_frameColor}{rgb}{0.75,0.75,0.75}
\definecolor{haskell_codeColor}{rgb}{0.86,0.85,0.89}

\iflsthaskell
  \usepackage{float}
\usepackage{listings}
\usepackage[scaled]{beramono}

\makeatletter
\newcommand\fs@plainruled{\def\@fs@cfont{\rmfamily}\let\@fs@capt\floatc@plain%
  \def\@fs@pre{\hrule\kern2pt}%
  \def\@fs@mid{\kern2pt\hrule\vspace\abovecaptionskip\relax}%
  \def\@fs@post{}%
  \let\@fs@iftopcapt\iffalse}
\makeatother

\floatstyle{plainruled}

\newfloat{listingfloat}{tp}{lop}
\floatname{listingfloat}{\bf Figure}

\lstloadlanguages{Haskell}

\lstset{
    language=Haskell,
    keywords={where,let,in,if,then,else,case,of,do,type,data,class,instance,family
             ,newtype,deriving,import},
    otherkeywords={::},
    morecomment=[l]{--},
    literate=
        {>=}{{$\geq$}}2
        {<=}{{$\leq$}}2
        {\\}{{$\lambda$}}1
        {\\\\}{{\char`\\\char`\\}}1
        {->}{{$\rightarrow$}}2
        {<-}{{$\leftarrow$}}2
        {=>}{{$\Rightarrow$}}2
        {<>}{{$\circ$}}1
        {>>=}{{>>=}}3
        {>>}{{>{>}}}2
        {===>}{{$\Longrightarrow$}}3
        {|->}{{$\mapsto$}}2
        {/\\}{{$\cap$}}2
        {forall}{{$\forall$}}0
        {undefined}{{???}}3
        {_=()}{{$\ldots$}}3
        {OLD}{}0
        {NEW}{}0
        {BAD}{}0
        {¤}{}0
        {ZERO}{{$_{\text{0}}$}}1
        {ONE}{{$_{\text{1}}$}}1
        {TWO}{{$_{\text{2}}$}}1
        {THREE}{{$_{\text{3}}$}}1
        {FOUR}{{$_{\text{4}}$}}1
        {FIVE}{{$_{\text{5}}$}}1
        {_PLUS}{{$_+$}}1
        {_TRIANG}{{$_\triangle$}}1
        {_DIAMOND}{{$_\diamond$}}1
        {_FO}{{$_{\text{FO}}$}}2
        {_HO}{{$_{\text{HO}}$}}2
        {\\/}{{$\sqcup$}}2
        {prop_}{{$\forall$}}1
        {=id$}{{$.$}}1
        {_NS}{{$_{\text{NS}}$}}2
        {_SPEC}{{$_{\text{SPEC}}$}}3
        {_CIRC}{{$_{\text{CIRC}}$}}3
        {>>>>}{{$>$}}2
        {_CXT}{{$_{\text{CXT}}$}}3
        {PROOF}{}0
        {_V}{{$_\text{V}$}}1
        {_BV}{{$_\text{BV}$}}2
        {-->}{{$\longrightarrow$}}4
}

\ifbeamer
\else
  \lstnewenvironment{code}{\minipage{\linewidth}}{\endminipage}
    % Already set by Pandoc's default Beamer template. There doesn't seem to be a way to renew
    % listings environments. But this environment is only needed when using Pandoc with
    % `-t beamer+lhs`, so the solution is to use `-t beamer`.
\fi

\lstnewenvironment{codex}{\minipage{\linewidth}}{\endminipage}
\lstnewenvironment{ghci}{\minipage{\linewidth}\lstset{keywords={Main}}}{\endminipage}

%\lstnewenvironment{code}{\lstset{basicstyle=\fontsize{8}{10}\fontencoding{T1}\ttfamily}}{}
%\lstnewenvironment{codex}{\lstset{basicstyle=\fontsize{8}{10}\fontencoding{T1}\ttfamily}}{}
%\lstnewenvironment{ghci}{\lstset{basicstyle=\fontsize{8}{10}\fontencoding{T1}\ttfamily}\lstset{keywords={Main}}}{}


  \ifbeamer

    \lstset{
        basicstyle=\fontsize{8}{10}\fontencoding{T1}\ttfamily,
        breaklines=false,
        breakatwhitespace=false,
        xleftmargin=0.3em,
        keepspaces=true,
        % columns=flexible,
        showstringspaces=false,
        frame=single,
        rulecolor=\color{haskell_frameColor},
        backgroundcolor=\color{haskell_codeColor}
    }

  \else

    \lstset{
        basicstyle=\fontsize{8}{10}\fontencoding{T1}\ttfamily,
        breaklines=false,
        breakatwhitespace=false,
        xleftmargin=0.9em,
        keepspaces=true,
        % columns=flexible,
        showstringspaces=false
    }

  \fi
\fi


%%%%%%%%%%%%%%%%%%%%%%%%%%%%%%%%%%%%%%%%%%%%%%%%%%%%%%%%%%%%%%%%%%%%%%%%%%%%%%%%%%%%%%%%%%%%%%%%%%%%

\newcommand{\ignore}[1]{}

\newcommand{\todo}[1]{~\newline\textcolor{red}{\underline{TODO}: {\em #1}}\newline}

\lstdefinelanguage{JavaScript}{
  keywords={break, case, catch, continue, debugger, default, delete, do, else, finally, for, function, if, in, instanceof, new, return, switch, this, throw, try, typeof, var, void, while, with},
  morecomment=[l]{//},
  morecomment=[s]{/*}{*/},
  morestring=[b]',
  morestring=[b]",
  sensitive=true
}


\begin{document}
\special{papersize=8.5in,11in}
\setlength{\pdfpageheight}{\paperheight}
\setlength{\pdfpagewidth}{\paperwidth}

\exclusivelicense
\conferenceinfo{IFL~'15}{September 14--16, 2015, Koblenz, Germany} 
\copyrightyear{2015}
\copyrightdata{bogus}
\doi{bogus}

% Uncomment one of the following two, if you are not going for the 
% traditional copyright transfer agreement.


%\permissiontopublish             % ACM gets nonexclusive license to publish
                                  % (paid open-access papers, 
                                  % short abstracts)

\titlebanner{-- IFL '15 draft submission --}        % These are ignored unless
\preprintfooter{-- -- -- -- draft -- -- -- --}   % 'preprint' option specified.

\title{Foreign Exchange at Low, Low Rates}
\subtitle{A lightweight FFI for web-targeting Haskell dialects}

\authorinfo{Anton Ekblad}
           {Chalmers University of Technology}
           {antonek@chalmers.se}

\maketitle

\begin{abstract}
  We present a novel yet simple foreign function interface, designed for
  web-targeting Haskell dialects but also applicable to a wider range of
  high-level target languages.
  The interface automates marshalling, eliminates boilerplate
  code, allows increased sanity checking of external data,
  allows the import of functions as well as arbitrary expressions of host
  language code, and is implementable as a plain Haskell '98 library
  without any modification to the Haskell compiler or environment.
  
  We give an implementation of this interface for the JavaScript-targeting
  Haste compiler, and show how the basic implementation can be
  further optimized with minimal effort to perform on par with Haskell's
  vanilla foreign function interface, as well as extended to support automatic
  marshalling of higher-order functions and automatic marshalling of host
  language exceptions.
  We also discuss how the interface can be implemented across a larger range
  of host environments and target languages at the cost of a slight increase in
  implementation complexity.
\end{abstract}

\category{D.1.1}{Programming Techniques}{Applicative (functional) Programming}
\category{D.2.11}{Software Architectures}{Languages}
\category{D.2.12}{Interoperability}{Interface definition languages}

% general terms are not compulsory anymore, 
% you may leave them out
%\terms
%term1, term2
%
\keywords
compilers; interoperability; web

\section{Introduction}\label{sec:intro}
Interfacing with other languages is one of the more painful aspects of modern
day Haskell development. Consider figure \ref{lst:bad-ffi}, taken from the
standard libraries of GHC; a piece of code to retrieve the
current time\ \cite{time}. A relatively simple task, yet its implementation is
surprisingly complex.

\begin{listingfloat}
\begin{code}
data CTimeval = MkCTimeval CLong CLong

instance Storable CTimeval where
	sizeOf _ = (sizeOf (undefined :: CLong)) * 2
	alignment _ = alignment (undefined :: CLong)
	peek p = do
		s   <- peekElemOff (castPtr p) 0
		mus <- peekElemOff (castPtr p) 1
		return (MkCTimeval s mus)
	poke p (MkCTimeval s mus) = do
		pokeElemOff (castPtr p) 0 s
		pokeElemOff (castPtr p) 1 mus

foreign import stdcall unsafe "time.h gettimeofday"
   gettimeofday :: Ptr CTimeval -> Ptr () -> IO CInt

getCTimeval :: IO CTimeval
getCTimeval = with (MkCTimeval 0 0) $ \ptval -> do
  throwErrnoIfMinus1_ "gettimeofday" $ do
    gettimeofday ptval nullPtr
  peek ptval
\end{code}
\caption{Foreign imports using the vanilla Foreign Function Interface}
\label{lst:bad-ffi}
\end{listingfloat}

This code snippet is more akin to thinly veiled C code than idiomatic, readable
Haskell; an unfortunate reality of working with the standard foreign function
interface. When using compilers such as Haste\ \cite{haste} and
GHCJS\ \cite{ghcjs} -- Haskell compilers which target the web browser -- the
situation is even worse.
The modern web browser environment is highly reliant on callback functions and
complex data types, none of which are trivial to pass through the FFI;
the user has a wealth of high-level JavaScript libraries within arm's reach,
but is forced to go through the low-level gateway of
the Haskell FFI\ \cite{ffi} to touch them.
While the example given in figure \ref{lst:bad-ffi} certainly works when
compiled with either Haste or GHCJS, it is not something the user would like
to write.

Traditionally, Haskell programs have used the Foreign Function Interface
extension to communicate with other languages. This works
passably well in the world of native binary programs running on bare metal,
where C calling conventions have become the de facto standard of foreign
data interchange.
The C language has no notion of higher-level data structures or
fancy data representation, making it the perfect lowest common denominator
interlingua for language to language communication:
there is no ambiguity or clash between different languages' built-in
representation of various higher-level data structures, as there simply
\emph{are} no higher-level data structures on the interface level.

The same properties that make Haskell's traditional foreign function interface
a good fit for language interoperability make it undesirable as a vehicle for
interfacing with the web-targeting code produced by compilers such as Haste
and GHCJS: said Haskell implementations commonly rely on the browser environment
for a large part of their runtime and internally use many of its native
high-level data structures and representations, making the forced low-level
representations of the vanilla foreign function interface an unnecessary
obstacle rather than a welcome common ground for data interchange.

With this background, we believe that low-level interfaces such as the
vanilla FFI are not ideally suited to the domain of functional languages
targeting the web browser or other high-level environments.
More specifically, we would like a foreign function interface for this domain
to have the following properties:

\begin{itemize}
\item
  The FFI should automatically take care of marshalling for any types where
  marshalling is defined, without extra manual conversions or other
  boilerplate code.
\item
  Users should be able to easily define their own marshalling schemes for
  arbitrary types.
\item
  The FFI should allow importing arbitrary snippets of foreign code, not just
  named, statically known functions. This allows users to efficiently compose
  code from different libraries in a single import,
  as well as transform data which may be inefficient or cumbersome to import
  into Haskell as-is.
\item
  Finally, the FFI should be easy to implement and understand, ideally being
  implementable without compiler modifications, portable across Haskell
  implementations targeting high-level environments.
\end{itemize}

Making this list a bit more concrete in the form of an example, we would like
to write high level code like that in figure \ref{lst:good-ffi}, without
having to make intrusive changes to our Haskell compiler.

\begin{listingfloat}
\begin{code}
data UTCTime = UTCTime {
    secs  :: Word,
    usecs :: Word
  } deriving Generic

instance FromAny UTCTime

getCurrentTime :: IO UTCTime
getCurrentTime =
  host "() => {var ms = new Date().getTime();\
              \return {secs:  ms/1000,\
              \        usecs: (ms % 1000)*1000};}"
\end{code}
\caption{Foreign imports using our FFI}
\label{lst:good-ffi}
\end{listingfloat}

Contrasting this with the standard FFI code from figure \ref{lst:bad-ffi}:

\begin{itemize}
\item
  The low-level machine types are gone, replaced by a more descriptive record
  type, and so is the \lstinline!peek!ing and \lstinline!poke!ing of pointers.
\item
  The imported function arrives ``batteries included'', on equal footing with
  every other function in our program. No extra scaffolding or boilerplate
  code is necessary.
\item
  Whereas the code in figure \ref{lst:bad-ffi} had to import the
  \lstinline!gettimeofday! system call by name, its actual implementation
  given elsewhere, we have actually \emph{implemented} its JavaScript
  counterpart at the location of its import from the building blocks
  available to us, without having to resort to external stubs.
\end{itemize}

To be clear, the idea of a higher level foreign function interface is by no
means novel in itself;
there already exists a large body of work in this problem domain, solving many
of the problems of figure \ref{lst:bad-ffi},
which is used here as an example mainly to
establish the baseline for foreign function interfaces.

We discuss these related approaches in section \ref{sec:related}, contrasting
them with our approach. To our knowledge, our solution is the first to address
all of the aforementioned criteria however.
In particular, we are not aware of any other FFI framework that can be
implemented entirely without compiler modifications.

\paragraph{Our contribution}\label{sec:contrib}
In section \ref{sec:interface}, we present a novel interface for a web-targeting
Haskell dialect to interface with its JavaScript host environment at a high level of
abstraction, and describe its implementation for the Haste compiler.
The interface lets users import arbitrary host language expressions in addition
to the named functions traditionally importable through Haskell's FFI.
This enables users to create efficient bindings to foreign code with a potentially
high impedance mismatch without having to pay with excessive boilerplate code.
It allows for context dependent sanity checking of incoming data from the host
language, improving the safety of foreign functions.

The interface makes use of dynamic code evaluation and the fact that
JavaScript -- the ``machine language'' of the web -- is intended for human
consumption to achieve a surprisingly lightweight implementation, which
\emph{does not rely on modifications to the Haskell compiler}; a feat which,
to our knowledge, we are the first to perform.

The basic interface is implementable using plain Haskell '98 with the
Foreign Function Interface extension, and is extensible by the user in the
types of data which can be marshalled between Haskell and host language,
as well as in how those types are marshalled.

In section \ref{sec:optimizations} we discuss various safety and performance
concerns about our implementation, and show how these concerns can be
alleviated by reaching outside the confines of Haskell '98.

In section \ref{sec:extensions} we show the flexibility of our design by using
it to implement marshalling of higher-order functions between the Haskell and
JavaScript, as well as a mechanism for automatically marshalling JavaScript
exceptions into Haskell equivalents.
We also discuss how to remove dynamic code evaluation from the equation
with a slight modification to the Haskell compiler in use.

\section{An FFI for the modern web}\label{sec:interface}
\subsection{The interface}
This section describes the programmer's view of our interface and gives
examples of its usage.
The Haskell formulation of the interface is given in figure \ref{lst:interface}.

As the main purpose of a foreign interface is to shovel data back and
forth through a rift spanning two separate programming worlds, it makes sense
to begin the description of any such interface with one central question:
what data can pass through the rift and come out on the other side still making
sense?

The class of data fulfilling this criterion is embodied in an
abstract \lstinline!HostAny! data type, inhabited by host-native
representations of arbitrary Haskell values.
A data type is then considered to be marshallable if and only if it can be
converted to \lstinline!HostAny! and back again.

Having established the class of types that can be marshalled, we can now give
a meaningful definition of \emph{importable} functions: a function can be
imported from the host language into our Haskell program if and only if:
\begin{itemize}
\item
  all of its argument types are convertible into \lstinline!HostAny!;
\item
  its return type is convertible \emph{from} the host-native
  \lstinline!HostAny!; and
\item
  its return type resides in the \lstinline!IO! monad, accounting for the
  possibility of side effects in host language functions.
\end{itemize}

\begin{listingfloat}
\begin{code}
type HostAny

class ToAny a where
  toAny :: a -> HostAny
  default toAny :: (GToAny (Rep a), Generic a)
                => a -> HostAny

class FromAny a where
  fromAny :: HostAny -> IO a
  default fromAny :: (GFromAny (Rep a), Generic a)
                  => HostAny -> IO a

class Import f
instance (ToAny a, Import b) => Import (a -> b)
instance FromAny a           => Import (IO a)

-- Instances for functions and basic types
instance ToAny Int
instance FromAny Int
...
instance Import f => FromAny f
instance (FromAny a, Exportable b) => ToAny (a -> b)
instance ToAny a => ToAny (IO a)

host :: Import f => String -> f
\end{code}
\caption{The programmer's view of our interface}
\label{lst:interface}
\end{listingfloat}

These definitions give rise to a workflow for interacting with host
language code:
\begin{itemize}
\item
  define the appropriate \lstinline!ToAny! and \lstinline!FromAny! instances
  for any custom types, either automatically using the generic default instances
  as showcased by our motivating example in figure \ref{lst:good-ffi},
  or by defining them manually if a particular host language representation is
  desired; then
\item
  import arbitrary host language symbols or expressions over
  any set of types instantiating \lstinline!ToAny! and \lstinline!FromAny!,
  using the \lstinline!host! function.
\end{itemize}

At first glance, it might seem strange to separate \lstinline!ToAny!
and \lstinline!FromAny! instead of merging them into a single
\lstinline!Marshal! class. The reason for this is that merging the two classes
breaks marshalling of pure higher order functions in a rather subtle way, as
discussed in section \ref{sec:extensions}.

We let the classic ``hello, world'' example illustrate the import of simple
host language functions using the interface described in figure
\ref{lst:interface}:

\begin{code}
hello :: String -> IO ()
hello = host "name => alert('Hello, ' + name);"
\end{code}

To further illustrate how this interface can be used to effortlessly import
even higher-order foreign functions, we have used our library to
implement bindings to JavaScript \emph{animation frames} for the Haste
compiler, a mechanism whereby a user program may request the browser to call
a certain function before the next repaint of the screen occurs:

\begin{code}
type Time = Double
newtype FrameHandle = FrameHandle HostAny
  deriving FromAny

requestFrame :: (Time -> IO ()) -> IO FrameHandle
requestFrame = host "window.requestAnimationFrame"

cancelFrame :: FrameHandle -> IO ()
cancelFrame = host "window.cancelAnimationFrame"
\end{code}

The resulting code is straightforward and simple, even though it performs
the rather non-trivial task of importing a foreign higher-order function,
automatically converting user-provided Haskell callbacks to their JavaScript
equivalents.

In the rest of section \ref{sec:interface}, we give an implementation of the
basic Haskell '98 interface for the Haste compiler.
We then extend it with features requiring some extensions to Haskell '98 --
most notably generics and default signatures -- in section
\ref{sec:extensions}, to arrive at the complete interface presented here.

\subsection{Implementing marshalling}\label{sec:marshalling}
As usual in the functional world, we ought to start with the \emph{base case}:
implementing marshalling for the basic primitive types that lie at the bottom
of every data structure.

This is a simple proposition, as this is the forte of the vanilla foreign
function interface.

\begin{code}
foreign import stdcall intToAny :: Int -> HostAny
foreign import stdcall anyToInt :: HostAny -> IO Int

instance ToAny   Int where toAny   = intToAny
instance FromAny Int where fromAny = anyToInt
\end{code}

We might also find a \lstinline!HostAny! instance for \lstinline!ToAny!
and \lstinline!FromAny! handy. Of course, \lstinline!HostAny! already being
in its host language representation form, the instances are trivial.

\begin{code}
instance ToAny   HostAny where toAny   = id
instance FromAny HostAny where fromAny = return
\end{code}

However, if passing simple values was all we wanted to do, then there would be
no need to look any further than the vanilla foreign function interface.
We must also provide some way of combining values into more complex values,
to be able to represent lists, record types and other conveniences we take for
granted in our day to day development work.
But how should these values be combined? Depending on our host language, we
may have different primitive data structures at our disposal.

Fortunately, JavaScript, as well as virtually any other
language for which our interface is implementable as described in section
\ref{sec:contrib}, support two basic aggregate types, which are sufficient to
represent values of any type: arrays and dictionaries.

For the sake of brevity, we assume that we have
access to a pair of functions
\lstinline!arrToList :: ToAny a => HostAny -> IO [a]!
and \lstinline!listToArr :: FromAny a => [a] -> HostAny!
which are used to implement the marshalling instance for lists;
they are trivial to implement either in Haskell using the vanilla foreign
function interface to gradually build a list of \lstinline!HostAny! values,
or on the host language side exploiting knowledge of the compiler's
data representation.

For dictionaries, the conversion is not as clear-cut. Depending on the data
we want to convert, the structure of our desired host language representation
of two values may well be different even when their Haskell
representations are quite similar, or even identical.
Hence, we need to put the power over this decision into the hands of the user,
providing functionality to build as well as inspect user-defined dictionaries.

We will need three basic host language operations: creating a new dictionary,
associating a dictionary key with a particular value, and looking up values
from dictionary keys. From these we construct two functions to marshal compound
Haskell values to and from dictionaries: \lstinline!mkDict!
and \lstinline!getMember!, as shown in figure \ref{lst:dicts}.
While a \lstinline!Map! would normally be the go-to data structure for
describing dictionaries, \lstinline!mkDict! only iterates over its argument
in order to add the corresponding entries to the created JavaScript dictionary,
making simple association lists a less heavyweight choice than a
\lstinline!Map!.

Note also the use of the generally unsafe \lstinline!unsafePerformIO!
in \lstinline!mkDict!. The only side effects performed by
\lstinline!mkDict! are to create a new reference, mutate it, and then
return it, never to mutate it again. As the reference to the mutated object
is not accessible outside \lstinline!mkDict! until after all mutation has
taken place, these side effects are not observable and this use of
\lstinline!unsafePerformIO! can thus be considered safe.

\begin{listingfloat}
\begin{code}
foreign import stdcall
  newDict :: IO HostAny
foreign import stdcall
  set :: HostAny -> HostString -> HostAny -> IO ()
foreign import stdcall
  get :: HostAny -> HostString -> IO HostAny

mkDict :: [(String, HostAny)] -> HostAny
mkDict xs = unsafePerformIO $ do
  d <- newDict
  mapM_ (\(k, v) -> set d (toHostString k) v) xs
  return d

getMember :: FromAny a => HostAny -> String -> IO a
getMember dict key =
  get dict (toHostString key) >>= fromAny
\end{code}
\caption{Dictionary manipulation}
\label{lst:dicts}
\end{listingfloat}

This gives us the power to represent any composite or primitive data
type with user-defined dictionary keys. Figure \ref{lst:sumprod} shows a
possible marshalling for sum and product types using the aforementioned
dictionary operations.

\begin{listingfloat}
\begin{code}
instance (ToAny a, ToAny b) =>
         ToAny (Either a b) where
  toAny (Left a)  = mkDict [("tag",  toHostString "left"),
                            ("data", toAny a)]
  toAny (Right b) = mkDict [("tag",  toHostString "right"),
                            ("data", toAny b)]

instance (FromAny a, FromAny b) =>
         FromAny (Either a b) where
  fromAny x = do
    tag <- fromHostString <$> getMember x "tag"
    case tag of
      "left"  -> Left  <$> getMember x "data"
      "right" -> Right <$> getMember x "data"

instance (ToAny a, ToAny b) => ToAny (a, b) where
  toAny (a, b) = toAny [toAny a, toAny b]

instance (FromAny a, FromAny b) => FromAny (a, b) where
  fromAny x = do
    [a, b] <- fromAny x
    (,) <$> fromAny a <*> fromAny b
\end{code}
\caption{Sums and products using lists and dictionaries}
\label{lst:sumprod}
\end{listingfloat}

It is worth noting that the implementation of \lstinline!getMember! is
the reason for \lstinline!fromAny! returning a value in the \lstinline!IO!
monad: foreign data structures are rarely, if ever, guaranteed to be immutable
and looking up a key in a dictionary is effectively following a reference,
so we must perform any such lookups at a well-defined point in time,
lest we run the risk of the value being changed in between the application
of our marshalling function and the evaluation of the resulting thunk.

\subsection{Importing functions}\label{sec:import}
Implementing the \lstinline!host! function turns out to be slightly trickier
than marshalling data between environments.
The types of our imported functions need to differ depending on the arity of
the imported host language code. This necessitates \lstinline!host! returning
some variadic function. Fortunately, there is a well known trick to accomplish
this, described in\ \cite{printf}, which uses an inductive class instance to
successively build up a list of arguments over repeated function applications,
and a base case instance to perform some computation over said arguments after
the function in question has been fully applied.
In the case of the \lstinline!host! function, that computation would be
applying a foreign function to said list of arguments.

This suggests the following class definition.

\begin{code}
type HostFun = HostAny
class Import f where
  import_ :: HostFun -> [HostAny] -> f
\end{code}

For our purposes, the base case is a nullary computation in the IO monad.
The list of arguments is converted from a list to
a host language array in order to squeeze it through the vanilla foreign
function interface, and the value we get back is marshalled back into a proper
Haskell value:

\begin{code}
foreign import stdcall
  apply :: HostFun -> HostAny -> IO HostAny

instance FromAny a => Import (IO a) where
  import_ f args =
    apply f (toAny args) >>= fromAny
\end{code}

Note the use of a foreign import in our base case. As the application of a
foreign function to a foreign list of foreign arguments is clearly, well, a
\emph{foreign} matter, we must call out to the host language for this final
step.
This function is specific to the host language in use. A possible
implementation of \lstinline!apply! for a JavaScript host environment may look
as follows:

\begin{code}
(f, args) => f.apply(null, args)
\end{code}

The inductive case is not much more complex: we only need to marshal a single
argument and recurse.

\begin{code}
instance (ToAny a, Import b) =>
         Import (a -> b) where
  import_ f args =
    \arg -> import_ f (toAny arg : args)
\end{code}

With this, we have all the building blocks required to implement the
\lstinline!host! function. With all the hard work already done, the
implementation is simple. For the sake of brevity, we assume the
existence of a host language specific \lstinline!HostString! type, which may
be passed as an argument over the vanilla foreign function interface, and a
function \lstinline!toHostString :: String -> HostString!.

\begin{code}
foreign import stdcall
  eval :: HostString -> HostFun

host :: Import f => String -> f
host s = import_ f []
  where
    f = eval (toHostString s)
\end{code}

The foreign \lstinline!eval! import brings in the host language's evaluation
construct. Recall that one requirement of our method is the existence of such
a construct, to convert arbitrary strings of host language code into functions
or other objects.

\section{Optimizing for safety and performance}
\label{sec:optimizations}
While the implementation described up until this point is more or less
feature complete, its non-functional properties can be improved quite a bit
if we allow ourselves to stray from the tried and true, but slightly
conservative, path of pure Haskell '98.

Aside from implementation specific tricks -- exploiting knowledge about a
particular compiler's data representation to optimize marshalling, or even
completely unroll and eliminate some of the basic interface's primitive
operations, for instance -- there are several general optimizations we can
apply to significantly enhance the performance and safety of our interface.

\subsection{Eliminating argument passing overheads}
\label{sec:specialization}
The performance-minded reader may notice something troubling about the
implementation of \lstinline!import_!: the construction of an intermediate list
of arguments. Constructing this intermediate list only to convert it into a
host language suitable representation which is promptly deconstructed as soon
as it reaches the imported function takes a lot of work. Even worse, this work
does not provide any benefit for the task to be performed: applying a foreign
function.

By the power of \emph{rewrite rules}\ \cite{rewriterules}, we can
eliminate this pointless work in most cases by specializing the
\lstinline!host! function's base case instance for different numbers of
arguments.
In addition to the general \lstinline!apply! function we define a series of
\lstinline!apply0!, \lstinline!apply1!, etc. functions, one for each arity
we want to optimize function application for.
The actual specialization is then a matter of
rewriting \lstinline!host! calls to use the appropriate application function.

Figure \ref{lst:specialization} gives a new implementation of the base case
of the \lstinline!Import! class which includes this optimization, replacing
the one given in section \ref{sec:interface}.

\begin{listingfloat}
\begin{code}
{-# NOINLINE [0] dispatch #-}
dispatch :: FromAny a => HostFun -> [HostAny] -> IO a
dispatch f args = apply f (toAny args) >>= fromAny

instance FromAny a => Import (IO a) where
  import_ = dispatch

foreign import stdcall apply0 ::
  HostFun -> IO HostAny
foreign import stdcall apply1 ::
  HostFun -> HostAny -> IO HostAny
foreign import stdcall apply2 ::
  HostFun -> HostAny -> HostAny -> IO HostAny
...

{-# RULES
"apply0" [1] forall f. dispatch f [] =
               apply0 f >>= fromAny
"apply1" [1] forall f a. dispatch f [a] =
               apply1 f a >>= fromAny
"apply2" [1] forall f a b. dispatch f [b,a] =
               apply2 f a b >>= fromAny
...
 #-}
\end{code}
\caption{Specializing the \lstinline!host! base case}
\label{lst:specialization}
\end{listingfloat}

\subsection{Preventing code injection}
\label{sec:safe_host}
Meanwhile, the \emph{safety-conscious} reader may instead be bristling at the
thought of executing code contained in something as egregiously untyped and
untrustworthy as a common string.
Indeed, by allowing the conversion of arbitrary strings
into functions, we're setting ourselves up for cross-site scripting and other
similar code injection attacks!

While this is indeed true in theory, in practice, accidentally passing a
user-supplied string to the \lstinline!host! function, which in normal use
ought to occur almost exclusively on the top level of a module, is a quite
unlikely proposition. Even so, it could be argued that if it is possible to use
an interface for evil, its users almost certainly will at some point.

Fortunately, the recent 7.10 release of the GHC compiler gives us the means to
eliminate this potential pitfall. The \emph{StaticPointers} extension, its
first incarnation described in\ \cite{cloudhaskell}, introduces the
\lstinline!static! keyword, which is used to create values of type
\lstinline!StaticPtr! from closed expressions. Attempting to turn any
expression which is not known at compile time into a \lstinline!StaticPtr!
yields a compiler error.

Implementing a \lstinline!safe_host! function which can not be used
to execute user-provided code becomes quite easy using
this extension and the basic \lstinline!host! function described in section
\ref{sec:interface}, at the cost of slightly more inconvenient import syntax:

\begin{code}
safe_host :: Import f => StaticPtr String -> f
safe_host = host . deRefStaticPtr

safe_hello :: IO ()
safe_hello = safe_host $
  static "() => alert('Hello, world!')"
\end{code}

\subsection{Eliminating \lstinline!eval!}
\label{sec:inline-eval}
Relying on \lstinline!eval! to produce our functions allows us to
implement our interface in pure Haskell '98 without modifying the Haskell
compiler in question, making the interface easy to understand, implement and
maintain. However, there are reasons why it may be in the implementor's best
interest to forgo a small bit of that simplicity.

The actual call to \lstinline!eval! does not meaningfully impact
performance: it is generally only called once per import,
the resulting function object cached thanks to lazy evaluation.\footnote{
  The main reason for \lstinline!eval! getting called more than once
  being unwise inlining directives from the user.
}
However, its dynamic nature \emph{does}
carry a significant risk of interfering with the ability of the host language's
compiler and runtime to analyse and optimize the resulting code.
As discussed in section \ref{sec:performance}, this effect is very much in
evidence when targeting the widely used V8 JavaScript engine.

In the JavaScript community, it is quite common to run programs through a
\emph{minifier} -- a static optimizer with focus on code size -- before
deployment. Not only do such optimizers suffer the same analytical
difficulties as the language runtime itself from the presence of dynamically
evaluated code, but due to the heavy use of
renaming often employed by minifiers to reduce code size, special care needs
to be taken when writing code that is not visible as such to the minifier:
code which is externally imported or, in our case, locked away inside a string
for later evaluation.

Noting that virtually every sane use of our interface evaluates a \emph{static}
string, a solution presents itself: whenever the \lstinline!eval! function
is applied to a statically known string, instead of generating a function call,
the compiler splices the contents of the string verbatim into the output code
instead.

This solution has the advantage of eliminating the code analysis obstacle
provided by \lstinline!eval! for the case when our imported code is statically
known (which, as we noted before, is a basic sanity property of foreign
imports), while preserving our library's simplicity of implementation.
However, it also has the \emph{dis}advantage of requiring modifications to the
compiler in use, however slight, which increases the interface's overall
complexity of implementation.

\section{Putting our interface to use}
\label{sec:extensions}
While the interface described in sections \ref{sec:interface} and
\ref{sec:optimizations} represents a clear raising of the abstraction layer
over the vanilla foreign function interface, it is still lacking some
desirable high level functionality: marshalling of arbitrary functions and
generic data types.

In this section we demonstrate the flexibility of our interface by implementing
this functionality on top of it.

\subsection{Dynamic function marshalling}
\label{sec:marshalling-functions}
\paragraph{Dynamic imports}
One appealing characteristic of our interface is that it makes the marshalling
of functions between Haskell and the host language easy. In the case of
passing host functions into Haskell, the \lstinline!import_! function used to
implement \lstinline!host! has already done the heavy lifting for us.
Only adding an appropriate \lstinline!FromAny! instance remains.

Due to the polymorphic nature of functions, however, we must resort to using
some language extensions to get the type checker to accept our instance:
overlapping instances, flexible instances, and undecidable instances.
Essentially, the loosened restrictions on type class instances allow an
\lstinline!Import! instance to act as a synonym for \lstinline!FromAny!,
allowing host language functions to return functions of any type admissible
as an import type by way of the \lstinline!host! function.

\begin{code}
instance Import a => FromAny a where
  fromAny f = return (import_ f [])
\end{code}

\paragraph{Passing functions to foreign code}
Passing functions the other way, out of Haskell and into our host language,
requires slightly more work. While we already had all the pieces of the
dynamic import puzzle at our disposal through our earlier implementation of
\lstinline!host!, exports require one more tool in our toolbox: a way to turn
a Haskell function into a native host language function.

Much like the \lstinline!apply! primitive used in the implementation of
\lstinline!host!, the implementation of such an operation is specific to the
host language in question. Moreover, as we are dealing with whatever format
our chosen compiler has opted to represent functions by, this operation is also
dependent on the compiler.

In order to implement this operation, we assume the existence of another
function \lstinline!hfsun_to_host!, to convert a Haskell function $f$ from $n$
\lstinline!HostAny! arguments to a \lstinline!HostAny! return value $r$ in the
IO monad into a host language function which, when applied to $n$ host
language arguments, calls $f$ with those same arguments and returns the $r$
returned by $f$.\\
\begin{code}
foreign import stdcall hsfun_to_host
  :: (HostAny -> ... -> HostAny) -> HostFun
\end{code}\\
But how can we make this operation type check?
As we are bound to the types the vanilla foreign function interface lets us
marshal, we have no way of applying this function to a variadic Haskell
function over \lstinline!HostAny!s.

We know that, operationally, \lstinline!hsfun_to_host! expects a Haskell
function as its input, but the types do not agree; we must somehow find a way
to pass arbitrary data unchanged to our host language.
Fortunately, standard Haskell provides us with a way to do exactly what we want:
\lstinline!StablePointer!s\ \cite{stableptr}. Note that, depending on the
Haskell compiler in use, this use of stable pointers may introduce a space
leak. This is discussed further in section \ref{sec:host-gc}, and an
alternative solution is presented.\\
\begin{code}
import Foreign.StablePtr
import System.IO.Unsafe

foreign import stdcall
  _hsfun_to_host :: StablePtr a -> HostFun

hsfun_to_host :: Exportable f => f -> IO HostFun
hsfun_to_host f =
  _hsfun_to_host `fmap` newStablePtr (mkHostFun f)
\end{code}\\
Just being able to pass Haskell functions verbatim to the host language is not
enough. The functions will expect Haskell values as their arguments and return
other Haskell values; we need to somehow modify these functions to
automatically marshal those arguments and return values.
Essentially, we want to map
\lstinline!fromAny! over all input arguments to a function, and
\lstinline!toAny! over its return values.
While superficially similar to the implementation of the \lstinline!Import!
class in section \ref{sec:import}, this task is slightly trickier:
where \lstinline!import_! modifies an arbitrary number of arguments and
performs some action with respect to a monomorphic value -- the
\lstinline!HostFun! representation of a host language function -- we now need
to do the same to a variadic function.

\paragraph{Modifying variadic functions using type families}
A straightforward application of the \lstinline!printf! trick used to
implement \lstinline!Import! is not flexible
enough to tackle this problem. Instead, we bring in yet another language
extension, closed type families\ \cite{closedtypefamilies}, to lend us the type
level flexibility we need. We begin by defining the \lstinline!Exportable!
type class first encountered in the type signature of \lstinline!hsfun_to_host!,
and a closed type family describing the type level behavior of our function
marshalling.\\
\begin{code}
type family Host a where
  Host (a -> b) = HostAny -> Host b
  Host (IO a)   = IO HostAny

class Exportable f where
  mkHostFun :: f -> Host f
\end{code}\\
This is relatively straightforward. Inspecting the \lstinline!Host! type
family, we see that applying \lstinline!mkHostFun! to any eligible function
must result in a corresponding function of the same arity -- hence the
recursive type family instance for \lstinline!a -> b! -- but with its
arguments and return value replaced by \lstinline!HostAny!.

Giving the relevant \lstinline!Exportable! instances is now mostly a matter of
making the types match up, and concocting a \lstinline!ToAny! instance is only
a matter of composing our building blocks together.

\begin{code}
instance ToAny a => Exportable (IO a) where
  mkHostFun = fmap toAny

instance (FromAny a, Exportable b) =>
          Exportable (a -> b) where
  mkHostFun f =
    mkHostFun . f . unsafePerformIO . fromAny
  
instance Exportable f => ToAny f where
  {-# NOINLINE toAny #-}
  toAny = unsafePerformIO . hsfun_to_host
\end{code}

The one interesting instance here is that of the inductive case, where we use
\lstinline!fromAny! in conjunction with \lstinline!unsafePerformIO! to marshal
a single function argument. While using \lstinline!fromAny! outside the
\lstinline!IO! monad is unsafe in the general case as explained in section
\ref{sec:interface}, this particular instance is completely safe, provided that
\lstinline!mkHostFun! is \emph{not} exported to the user, but only used to
implement the \lstinline!ToAny! instance for functions.

When a function is marshalled into a \lstinline!HostAny! value and subsequently
applied, \lstinline!fromAny! will be applied unsafely to each of the
marshalled function's arguments. There are two cases
when this can happen: either the marshalled function is called from the host
language, or it is marshalled back into Haskell and then applied.
In the former case, the time of the call is trivially well-defined assuming
that our target language is not lazy by default.
In the latter case, the time of the call is still well-defined, as our
interface only admits importing functions in the \lstinline!IO! monad.

Slightly more troubling is the use of \lstinline!unsafePerformIO! in
conjunction with \lstinline!hsfun_to_host!.
According to\ \cite{stableptr}, the creation of stable pointers residing in
the \lstinline!IO! monad -- the reason for \lstinline!hsfun_to_host! residing
there as well -- is to avoid accidentally duplicating the allocation of the
stable pointer, something we can avoid by telling the compiler never to
inline the function, ever.

It is also worth pointing out that the concern over duplicating this allocation
is only valid where the implementation also has the aforementioned space leak
problem, in which case the alternative implementation given in section
\ref{sec:host-gc} should be preferred anyway.

\paragraph{Marshalling pure functions}
The above implementation only allows us to pass functions in the \lstinline!IO!
monad to foreign code, but we would also like to support passing pure
functions. There are two main obstacles to this:

\begin{itemize}
\item
  The \lstinline!hsfun_to_host'! function expects a function in the
  \lstinline!IO! monad.
\item
  Instantiating \lstinline!Exportable! for any type \lstinline!ToAny t => t!
  would accidentally add a \lstinline!ToAny! instance for
  \emph{any type at all}.
  Even worse, this instance would be completely bogus for most types,
  always treating
  the argument to its \lstinline!toAny! implementation as a function to be
  converted into a host language function!
\end{itemize}

We sidestep the first problem by assuming that \lstinline!hsfun_to_host'!
can determine dynamically whether a function is pure or wrapped in the
\lstinline!IO! monad, and take action accordingly.
Another, slightly more verbose, possibility would be to alter the
implementation of our marshalling code to use either
\lstinline!hsfun_to_host'! or a function performing the same conversion
on pure functions, depending on the type of function being marshalled.

Looking closer at the problematic \lstinline!ToAny! instance, we find that
the \lstinline!Exportable t => ToAny t! instance provides
\lstinline!ToAny! for any \lstinline!Exportable! type, and the
\lstinline!ToAny t => Exportable t! instance provides \lstinline!Exportable! in
return, creating a loop which creates instances for both type classes
matching any type.

The \lstinline!ToAny t => Exportable t! instance is necessary
for our type level recursion to work out when marshalling pure functions,
but we can prevent this instance from leaking to \lstinline!ToAny! where it
would be unreasonably broad by replacing our \lstinline!ToAny! function
instance with two slightly more specific ones.

Figure \ref{lst:export-final} gives our final implementation of dynamic
function exports.
Looking at this code we also see why the use of closed type families are
necessary: the open type families originally introduced by Chakravarty et
al\ \cite{typefamilies} do not admit the overlapping type equations required
to make pure functions an instance of \lstinline!Exportable!.

\begin{listingfloat}
\begin{code}
import Foreign.StablePtr
import System.IO.Unsafe

foreign import stdcall
  _hsfun_to_host :: StablePtr a -> HostFun

hsfun_to_host :: Exportable f => f -> IO HostFun
hsfun_to_host f =
  _hsfun_to_host `fmap` newStablePtr (mkHostFun f)

type family Host a where
  Host (a -> b) = HostAny -> Host b
  Host (IO a)   = IO HostAny
  Host a        = HostAny

class Exportable f where
  mkHostFun :: f -> Host f

instance (ToAny a, Host a ~ HostAny) =>
          Exportable a where
  mkHostFun = toAny

instance (FromAny a, Exportable b) =>
          ToAny (a -> b) where
  toAny = unsafePerformIO . hsfun_to_host

instance ToAny a => ToAny (IO a) where
  toAny = unsafePerformIO . hsfun_to_host
\end{code}
\caption{Dynamic function exports implemented on top of our interface}
\label{lst:export-final}
\end{listingfloat}

\subsection{Static function exports}
\label{sec:exports}
Very rarely are users prepared to abandon person-decades of legacy code;
to reach these users, the ability to expose Haskell functionality to the host
language is important. Alas, being implemented as a library, our interface is
not capable of \lstinline!foreign export! declarations. We can, however,
implement a substitute on top of it.

Rather than a writing a library which when compiled produces a shared library
for consumption by a linker, we give the user access to a function
\lstinline!export! which when executed stores an exported function in a
known location, where foreign language code can then access it.
While this may seem like a silly workaround, this is how JavaScript programs
commonly ``link against'' third party libraries.

Using the function marshalling implemented in section
\ref{sec:marshalling-functions}, implementing \lstinline!export! becomes a
mere matter of passing a function to the host language, which then
arranges for the function to be available in a known, appropriate location.

\begin{code}
export :: ToAny f => String -> f -> IO ()
export =
  host "(name, f) => window['haskell'][name] = f;"
\end{code}

\subsection{Generic marshalling}
\label{sec:default-instances}
Returning to our motivating example with figure \ref{lst:good-ffi}, we note
a conspicuous absence: the \lstinline!UTCTime! instance of \lstinline!FromAny!
is not defined, yet it is still used by the \lstinline!host! function in the
definition of \lstinline!getCurrentTime!. Although the instance can be defined
in a single line of code, it would still be nice if we could avoid the tedium
of writing that one line altogether. As stated in section
\ref{sec:marshalling}, any non-arrow Haskell type can be represented using
a combination of arrays and dictionaries. Using one of the generic programming
frameworks offered by Haskell, such as GHC generics\ \cite{generics} or
Template Haskell\ \cite{th}, it is possible to create a \emph{default instance}
of the marshalling type classes, applicable to any Haskell type.

As the implementation of such an instance is neither novel nor particularly
interesting in the context of this paper, we refer the reader to the one
used by the \emph{aeson} package for encoding and decoding of
JSON values\ \cite{aeson}. This default instance provides the final piece
of the puzzle required to use the interface as presented in
figure \ref{lst:good-ffi}.


\subsection{Marshalling JavaScript exceptions}
\label{sec:marshalling-exceptions}
Trapping errors in foreign C code is relatively straightforward, albeit
cumbersome, owing to the relative absence of structured error handling in C.
However, when interacting with a higher level language, one must take into
account the risk of exceptions being raised in any imported foreign code.
In the basic Haste.Foreign interface, such exceptions must be manually handled
lest they terminate the enclosing Haskell program much like a segmentation
fault in imported C code would terminate a native Haskell program.

Thankfully, we can leverage the higher-order import capabilities described
in section \ref{sec:extensions} to catch JavaScript exceptions and re-throw
them within Haskell's exception handling framework.
We import a JavaScript function \lstinline!catchJS! which accepts an
exception handler function and an IO computation as its arguments.
When called, \lstinline!catchJS! executes IO computation inside a try-catch
block. If an exception is raised, it is passed to the exception handler
function which then takes appropraite action.

\lstinline!catchJS! may be used similarly to Haskell's \lstinline!catch!
function to catch exceptions in foreign functions where they are expected to
occur.
However, it is not necessarily the case that we always want to handle
exceptions right at the call site of a foreign function -- quite the opposite!
Instead, we can create an exception-safe equivalent to \lstinline!host! which
uses \lstinline!catchJS! to dispatch \emph{all} calls to functions imported
through it, with an exception handler function that simply re-throws the
JavaScript exception wrapped in a Haskell exception. The wrapped exception
can then be caught anywhere a ``normal'' Haskell exception could be caught.
A complete implementation of this extension is given in
figure \ref{lst:catchJS}.

\begin{listingfloat}
\begin{code}
catchJS :: (ToAny a, FromAny a)
        => (String -> IO ())
        -> IO a -> IO a
catchJS = "(handle, act) =>\\
  \\{ try        { return act(); }\\
  \\  catch (ex) { handle(ex.toString()); }}"

data HostException = HostException String
  deriving Show
instance Exception HostException
  
class Safely a where
  safely :: a -> a

instance Safely b => Safely (a -> b) where
  safely f x = safely (f x)

instance (FromAny a, ToAny a) => Safely (IO a) where
  safely m = catchJS (throwIO . HostException)

very_safe_host :: Import a => StaticPtr String -> a
very_safe_host = safely safe_host
\end{code}
\caption{Marshalling JavaScript exceptions}
\label{lst:catchJS}
\end{listingfloat}

It should be noted that this approach incurs a performance penalty due to the
extra function call and marshalling required to dispatch a function, as further
discussed in section \ref{sec:performance}.


\section{Performance}
\label{sec:performance}
While increased performance is not a major motivation for this work, it is
still important to ascertain that using our library does not entail a major
performance hit. To determine the runtime performance of our interface
vis a vis the vanilla FFI - a useful baseline for performance comparisons - we
have benchmarked a reference implementation of our interface against the
vanilla FFI, both implemented for the Haste compiler.

While benchmarking code outside the context of any particular application is
often tricky and not necessarily indicative of whole system performance,
we hope to give a general idea of how our library fares performance-wise in
several different scenarios.
To this end, several microbenchmarks were devised:

\begin{itemize}
\item
  \emph{Outbound}, which applies a foreign function to several arguments of
  type \lstinline!Double!. The function's return value is discarded, in order
  to only measure outbound marshalling overhead for primitive types.
\item
  \emph{In-out}, which applies a foreign function to several \lstinline!Double!
  arguments and marshals its return value, also of type \lstinline!Double!,
  back into Haskell land. This measures inbound as well as outbound marshalling
  of primitive types.
\item
  \emph{Product types}, which benchmarks the implementation of\linebreak
  \lstinline!getCurrentTime! given in figure \ref{lst:good-ffi} against
  the equivalent implementation given in figure \ref{lst:bad-ffi}, both
  modified to accept an \lstinline!UTCTime! value as input in addition to
  returning the current time, in order to measure outbound marshalling of
  product types as well as inbound.
\item
  \emph{HOF import}, which calls a higher-order function $f$ using
  both the vanilla FFI and our method, with a function over a
  single \lstinline!Double! value as its argument.
  The only purpose of $f$ is to call its argument repeatedly,
  evaluating the speed with which a higher-order Haskell function may be called
  from external code in addition to the speed of marshalling itself.
\end{itemize}

These functions were then applied 500 000 times in two different contexts: one
tight, strict, tail recursive loop, intended to produce as efficient code
as possible; and one which simply consists of running \lstinline!mapM_! over
a list containing 500 000 elements, to obtain higher level code which is harder to
optimize and analyse for strictness.

The resulting programs, compiled with version 0.5.4.2 of the Haste compiler
which incorporates all the optimizations described in section \ref{sec:optimizations},
were then repeatedly executed using version 4.2.2 of the Node.js JavaScript interpreter,
and the average run times of the programs using our interface compared against
the average run times of their FFI counterparts.
The benchmarks were executed on a Lenovo ThinkPad X230 laptop running Debian GNU/Linux,
equipped with an Intel i5 3210M CPU and 8 GB of RAM.

While this may not be the most rigorous of performance evaluations,
the results are repeatable, and the methodology is enough for our purposes:
getting a rough picture of how much speed we are giving up for a more
convenient interface.

The results for each benchmark are given in table \ref{tbl:performance} as
the ratio of the run time for our library over the run time for the vanilla
FFI.

\begin{table}
  \renewcommand{\arraystretch}{1.5}
  \begin{center}
    \begin{tabular}{|r|r|r|l|}
      \hline
                    & Tight loop & \lstinline!mapM_! & Tight + exceptions\\
      \hline
      Outbound      & 0.98       & 1.07              & 8.21\\
      \hline
      In-out        & 0.97       & 1.08              & 11.70\\
      \hline
      Product types & 0.83       & 0.95              & 2.42\\
      \hline
      HOF import    & 0.94       & 0.96              & 1.09\\
      \hline
    \end{tabular}
  \end{center}
  \caption{Execution times as fractions of FFI execution times}
  \label{tbl:performance}
\end{table}

\paragraph{Outbound}
Looking at the performance numbers, our library performs surprisingly well in
both the highly optimized and less optimized loop cases, with the loose loop
showing a modest 7 \% slowdown over the vanilla FFI, and the tight loop even
eking out a tiny performance benefit.

In contrast, the performance hit when using the exception-safe version defined
in section \ref{sec:marshalling-exceptions} is huge.
This is by no means surprising: using the exception-safe version entails
marshalling no less than two additional higher-order functions, which is quite
a bit heavier than the otherwise very lightweight marshalling performed for
plain numbers.

\paragraph{In-out}
Moving on to the benchmarks where we actually marshal incoming data, the
picture is much the same as for the \emph{outbound} benchmark. The performance
hit from the exception marshaller becomes even more problematic here, as the
return value of the function needs to be marshalled first into Haskell,
then back into JavaScript for the exception handler, and finally back into
Haskell again.

\paragraph{Product types}
Our interface shows a small performance advantage when it comes to
marshalling more complex values, being 5 -- 20 \% faster
depending on the loop. Our assumption about peeking and poking at pointers
being suboptimal in an environment where such operations are considerably
more expensive than on bare metal seems to have been correct.

Worth noting is that the performance overhead of the exception marshaller
becomes significantly less prohibitive in this benchmark, as the complexity
of marshalling grows. This indicates that while expensive compared to
the minimal marshalling required for base types, exception safety may not be
all that expensive after all, when calling a function which performs actual
work.

\paragraph{HOF import}
Our interface seems to compare favorably to the vanilla FFI for this case,
although the performance gain difference is quite minimal. This is to be
expected, as the heavy lifting required to export Haskell functions into
JavaScript is relatively similar and quite heavy regardless of how the
relatively lightweight marshalling of the function's base type arguments
is carried out. Again, it is worth pointing out how the significance of the
exception marshaller's performance penalty dwindles as the marshalling
process as a whole grows heavier.

\paragraph{Performance verdict: acceptable}
Judging by these numbers the performance of our library is quite acceptable,
with the exception of the heavy toll taken by the exception marshaller on
the less complex marshalling cases.
Interestingly, the optimization described in section \ref{sec:inline-eval}
does not impact performance measurably in these benchmarks. If any
performance benefit is to be had from this optimization, is will likely come
from increased opportunities for minifiers and JavaScript engines to perform
inlining over more complex programs.

It is encouraging that our interface's intended use case - marshalling
more complex types and higher-order functions - is showing tangible performance
benefits in addition to the added convenience it affords the user.
For code which has no choice but to make a large number of calls to
low level host language functions over primitive types in performance critical
loops, using the vanilla FFI instead, or at least handling JavaScript
exceptions in foreign code instead of relying on our library's exception
marshaller, may be an attractive option to
reduce the performance penalty incurred by our interface in unfavorable circumstances,
allowing the user to have the FFI cookie and eat it at the same time.

The benchmarks used here are available online from our repository at
\url{https://github.com/valderman/ffi-paper}.

\section{Discussion}\label{sec:discussion}
While two of the tree main limitations our interface places on its host
language -- the presence of a dynamic code evaluation construct and support for
first class functions -- have hopefully been adequately explained, and their
severity slightly alleviated, in sections \ref{sec:interface} and
\ref{sec:inline-eval}, there are still several design choices and lingering
limitations that may need further justification.

\subsection{fromAny type level expressiveness}
The \lstinline!fromAny! function used to implement marshalling in section
\ref{sec:interface} is by definition not total. As its purpose is to
convert values of god-knows-what host language type into properly typed Haskell
values, from the simplest atomic values to the most complex data structures,
the possibility for failure is apparent. Why, then, does its type not admit
the possibility of failure, for instance by wrapping the converted value in a
\lstinline!Maybe! or \lstinline!Either!?

Recall that \lstinline!fromAny! will almost always be called when
automatically converting arguments to and return values from callbacks and
imported foreign functions respectively. In this context, even if a conversion
were to fail with a \lstinline!Left "Bad conversion"! error, there is no way
for this error value to ever reach the user. The only sensible action for the
foreign call to take when encountering an error value would be to throw an
exception, informing the user ``out of band'' rather than by somehow threading
an error value to the entire call.

It is then simpler, as well as reducing the amount of error checking overhead
necessary, to trust that the foreign code in question is usually well behaved
and throw the previously mentioned exception immediately on conversion failure
rather than taking a detour via error values,
should this trust prove to be misplaced.

\subsection{Limitation to garbage collected host languages}
\label{sec:host-gc}
The observant reader may notice that up until this point, we have completely
ignored something which very much concerns traditional foreign function
interfaces: ownership and eventual deallocation of memory.

Our high level interface depends quite heavily on its target language being
garbage collected, as having to manually manage memory introduces significant
boilerplate code and complexity: the very things this interface aims to avoid.
As target platforms \emph{with} garbage collection having to deal with low
level details such as memory management is the core motivation for this work,
rectifying this issue does not fall within the scope of this paper.

Even so, memory management does rear its ugly head in section
\ref{sec:marshalling-functions}, where stable pointers are used to pass data
unchanged from Haskell into our host language, and is promptly ignored: note
the complete absence of calls to \lstinline!freeStablePtr!. Implementing our
interface for the Haste compiler, this is not an issue: Haste makes full use
of JavaScript's garbage collection capabilities to turn stable pointers into
fully garbage collected aliases of the objects pointed to.
It is, however, quite conceivable for an implementation to perform some manual
housekeeping of stable pointers even in a garbage collected language, in which
case this use of our interface will cause a space leak as nobody is keeping
track of all the stable pointers we create.

As the stable pointers in question are never dereferenced or otherwise
used within Haskell, this hypothetical space leak can be eliminated by
replacing stable pointers with a slight bit of unsafe, implementation-specific
magic.\\
\begin{code}
  import Unsafe.Coerce
  import Foreign.StablePtr hiding (newStablePtr)

  data FakeStablePtr a
  fakeStablePtr :: a -> FakeStablePtr a

  newStablePtr :: a -> StablePtr
  newStablePtr = unsafeCoerce . fakeStablePtr
\end{code}\\
The \lstinline!FakeStablePtr! type and the function by the same name are used
to mimic the underlying structure of \lstinline!StablePtr!. This makes its
exact implementation specific to the Haskell compiler in question, unlike
the ``proper'' solution based on actual stable pointers. The Haste
compiler, being based on GHC, has a very straightforward representation for
stable pointers, merely wrapping the ``machine'' level pointer in a single
layer of indirection, giving us the following implementation of fake stable
pointers:\\
\begin{code}
  data FakeStablePtr a = Fake !a

  fakeStablePtr = Fake
\end{code}\\
Thus, we may choose our implementation strategy depending on the capabilities
of our target compiler. For a single implementation targeting multiple
platforms however, proper stable pointers are the safer solution.

\subsection{Restricting imports to the IO monad}
The interface presented in this paper does not support importing pure
functions; any function originating in the host language must be safely locked
up within the IO monad. This may be seen as quite a drawback, as a host
language function operating solely over local state is definitely not
beyond the realms of possibility. Looking at our implementation of function
exports for pure functions, it seems that it would be possible to implement
imports in a similar way, and indeed we could.

However, ``could'' is not necessarily isomorphic to ``should''.
Foreign functions do, after all, come from the unregulated, disorderly world
outside the confines of the type checker. Haskell's type system does not
allow us to mix pure functions with possibly impure ones, and for good
reason. It is not clear that we should lift this
restriction just because a function is defined in another language.

Moreover, as explained in section \ref{sec:interface}, marshalling inbound
data is in many cases an inherently effectful operation, particularly when
involving complex data structures. Permitting the import of pure functions,
knowing fully well that a race condition exists in the time window between
the import's application and the resulting thunk's evaluation, does not strike
us as a shining example of safe API design.

Better, then, to let the user import their foreign code in the IO monad and
explicitly vouch for its purity, using \lstinline!unsafePerformIO! to bring it
into the world of pure functions.

\subsection{Continuation-based blocking in non-concurrent environments}
A particularly neat feature of the foreign function interface employed by the
GHCJS compiler is the ability for foreign host code to suspend execution while
waiting for an event to occur, even though its JavaScript host environment is
devoid of any concurrency support\ \cite{ghcjs}.
This is accomplished by giving imported functions an extra parameter:
a continuation to be called upon completion
of the foreign operation to resume execution of the Haskell program, instead
of simply returning like a ``normal'' JavaScript function would.

This functionality is not supported out of the box. GHCJS accomplishes this
by outputting continuation passing code which is executed by a clever
trampolining machinery. Supporting this feature would tie the interface to
a particular code generation strategy as well as add considerable complexity;
a price we deem too high to pay for this feature.

Instead, this functionality can be implemented on top of our interface without
much difficulty using a construct dubbed the ``poor man's concurrency
monad''\ \cite{poormansconcurrencymonad}; a monad implementing coarse grained
preemptive multitasking with blocking synchronization variables in
non-concurrent environments.

\subsection{Related work}\label{sec:related}
Aside from the vanilla foreign function interface used as the basis of our
interface, there are several different, more modern, takes on interfacing
purely functional languages with host language code.
One common denominator is specialization: without exception, these
implementations rely in large part on modifications to the compiler or
language itself, in contrast to our interface which makes some sacrifices
in order to be implementable as a library, maximizing portability across
host and Haskell implementations alike.

\paragraph{Idris: host-parametric FFI}
\emph{Idris} is a dependently typed, Haskell-like language with backends for
several host environments,\\
JavaScript being one of them\ \cite{idris}.
Like Haskell, Idris features monadic IO, but unlike Haskell, Idris' IO monad
\emph{is}, in a sense its foreign function interface.
IO computations are constructed from primitive building blocks, imported using
a function not unlike our \lstinline!host! function described in section
\ref{sec:interface}, and parameterized over the target environment.
This ensures that Idris code written specifically for a native environment
is not accidentally called from code targeting JavaScript and vice versa.

Idris' import function does not necessarily accept strings of foreign
language code, but is parameterized over the target environment just like the
IO monad; for JavaScript-targeting code, foreign code happens to be specified
as strings, but could conceivably consist of something more complex, such as
an embedded domain-specific language for building Idris-typed host language
functions.

\paragraph{Fay: featureful but static}
Our interface was partially inspired by the foreign function interface of the
\emph{Fay} language,
a ``proper subset of Haskell that compiles to JavaScript''\ \cite{fay}.
While the two are very similar in syntax, allowing users to import typed
strings of host language code, Fay's solution is highly specialized.
The compiler takes a heavy hand in the marshalling and import functionality,
parsing the host language code and performing certain substitutions on it.
While marshalling of arbitrary types is available, this marshalling is not
easily controllable by the user, but follows a sensible but fixed format
determined by the compiler.
This approach makes sense, as the interface is designed to support the Fay
language and compiler alone, but differs from our work which aims to create
a more generally applicable interface.

\paragraph{GHCJS: JavaScriptFFI}
The \emph{GHCJS} Haskell-to-JavaScript compiler\ \cite{ghcjs} utilizes the
relatively recent \emph{JavaScriptFFI} GHC extension,
which has unfortunately been rarely described outside a GHCJS context, to the
point of being conspicuously absent from even the GHC documentation.
Much like Fay, this extension parses and
performs substitutions over imported host language code to make imports
slightly more flexible, allowing for importing arbitrary expressions rather
than plain named functions.
It also enables additional safety levels for foreign
imports: \emph{safe}, where bad input data is replaced by default values and
foreign exceptions caught and marshalled into Haskell equivalents,
and \emph{interruptible}, which allows host language code to suspend execution
indefinitely even though JavaScript is completely single threaded.
This is accomplished by handing interruptible functions a continuation in
addition to their usual arguments, to call with the foreign function's
``return value'' as its argument when it is time for the foreign function
to return and let the Haskell program resume execution.

The JavaScriptFFI extension preserves the regular FFI's onerous restrictions
on marshallable types however, and while while GHCJS comes with convenience
functions to convert between these more complex types and the simple ones
allowed through the FFI, marshalling is not performed automatically and
functions in particular are cumbersome to push between Haskell and JavaScript.

\paragraph{UHC: the traditional FFI, on steroids for JavaScript}
The UHC Haskell compiler comes with a JavaScript backend as well, and matching
higher level extensions to its foreign function interface\ \cite{uhc}.
Like Fay, UHC provides automatic conversion of Haskell values to JavaScript
objects, as well as importing arbitrary JavaScript expressions, with some
parsing and wildcard expansion.
Also like Fay, the JavaScript representation produced by this
conversion is determined by the compiler, and is not user configurable.
UHC does, however, provide several low level primitives for manipulating
JavaScript objects from within Haskell, both destructively and in a purely
functional manner.

\paragraph{Clean: mixing host and client language code}
The Clean language sports a foreign function interface which differs slightly
from the rest of the interfaces discussed here. In Clean, the module system
makes a difference between \emph{definition modules}, where abstract types and
functions are declared, and \emph{implementation modules}, where implementations
are given for the types and functions declared in the corresponding definition
modules.
Instead of using a special ``foreign import'' syntactic form, Clean allows
developers to write \emph{system} implementation modules: modules where the
implementations of functions defined in a definition module may be written in
a language other than Clean\ \cite{clean}. However, only primitive types may
be passed to this foreign code and no guarantees, making higher level
interoperability cumbersome.
Clean's FFI is thus more flexible than the foreign function interface of GHC,
allowing host and client language code to be mixed, but less so than the other
interfaces discussed in this section due to its less expedient marshalling
capabilities.

\paragraph{Quasiquoting}
Quasiquoting represent another, more radically different, approach to the
problem of bridging with a host language\ \cite{quasiquotes}.
Allowing for the inline inclusion of large snippets of foreign code with
compile time parsing and type checking,
quasi-quotes have a lot in common with our interface, even eclipsing it in
power through anti-quotes, which allow the foreign code expressions to
incorporate Haskell data provided that the proper marshalling has been
implemented. Recent work by Manuel Chakravarty has extended the usefulness of
quasi-quotes even further, automating large parts of the stub generation
and marshalling required for using quasi-quoted host language code as a
foreign function interface\ \cite{language-c-inline}.

This usefulness comes at the price of a more involved implementation.
Quasiquoting requires explicit compiler support in the form of
compile time template metaprogramming as well as special extensions for running
the quasiquoters themselves.
In order to make full use of its compile time parsing and analysis capabilities
an implementor also need to supply a parser for the quoted language.

\section{Conclusions and future work}
\label{sec:conclusion}
\paragraph{Future work}
While our interface is designed for web-targeting Haskell dialects, extending
its applicability is generally a venue worthy of further exploration.

By combining two optimizations given in section \ref{sec:optimizations},
the restriction of our \lstinline!safe_host! function to only accept statically
known strings and the elimination of calls to \lstinline!eval! for statically
known strings, it is possible to remove the requirement that a potential
host language support dynamic code evaluation.
If all foreign imports are statically known,
and we are able to eliminate \lstinline!eval! calls for all statically known
functions, it follows that we are able to eliminate all \lstinline!eval! calls.
While the actual implementation of this idea has yet to be worked out,
guaranteeing the complete absence of \lstinline!eval! from the generated host
code would remove the restriction that our host language supports dynamic code
evaluation at runtime, notably making our interface implementable on
recent versions of the Java Virtual Machine.
Implementing this interface for the Java Virtual Machine, with the
prerequisite Haskell-to-JVM compiler, would lend additional applicability to
our interface.

Out interface does not currently catch and marshal host language
exceptions, but requires foreign language code to take care of any exceptions.
While the actual implementation is quite specific to a particular
host environment, automatically converting exceptions would be a useful feature
even so. Investigating the degree to which this feature could be implemented
in a host platform agnostic manner would be a possible extension of this work.

Due to the hard requirement that our host language be garbage collected,
our interface is not currently applicable in a C context. This is unfortunate,
as C-based host environments are still by far the most common for Haskell
programs. It may thus be worthwhile to investigate the compromises needed
to lift the garbage collection requirement from potential host environments.

\paragraph{Conclusions}
We have presented the design and implementation of a novel, portable foreign
function interface for web-targeting Haskell dialects. While designed for the
web sphere, the given implementation is also applicable to a wide range of
other high level target languages as well.

We have also given a number of optimizations, improving the performance and
safety of our interface
and lightening the restrictions placed on the host environment, and implemented
our interface as a library for the Haste Haskell-to-JavaScript compiler.
Finally, we have used this library to further extend our marshalling
capabilities to cover functions, as well as generic default marshalling for
arbitrary data types, contrasted our approach with a variety of existing
foreign function interfaces, and demonstrated that our library does not
introduce excessive performance overhead compared to the vanilla FFI.

While our interface is currently not applicable to Haskell implementations
targeting low level, C-like environments, it brings significant reductions in
boilerplate code and complexity for users needing to interface their Haskell
programs with their corresponding host environment in the space where it
\emph{is} applicable: Haskell implementations for high level target platforms.

Our reference implementation is shipped in both source code and binary form as
part of the Haste distribution, available from the Haste website
at\ \cite{haste-lang.org}.

\acks
This work has been partially funded by the Swedish Foundation for Strategic Research, under grant RAWFP. Many thanks to Koen Claessen, Emil Axelsson and Atze van der Ploeg for their valuable feedback and comments.

% We recommend abbrvnat bibliography style.

\bibliographystyle{abbrvnat}

% The bibliography should be embedded for final submission.

\begin{thebibliography}{}
\softraggedright

\bibitem[Augustsson, Massey (2013)]{printf}
  Augustsson, Lennart, and Massey, Bart. The {\tt Text.Printf} module.
  Available from \url{http://hackage.haskell.org/package/base-4.8.0.0/docs/Text-Printf.html},
  2015-04-01.
\bibitem[Balat, Chambart, Henry (2012)]{ocsigen}
  Balat, Vincent, Chambart, Pierre, and Henry, Grégoire.
  Client-server Web applications with Ocsigen.
  In WWW2012, p. 59. 2012.
\bibitem[Brady (2015)]{idris}
  Brady, Edwin.
  Cross-platform Compilers for Functional Languages.
  Under consideration for Trends in Functional Programming, 2015.
\bibitem[Chakravarty et al (2003)]{ffi}
  Chakravarty, M. M. (Ed.). (2003).
  The Haskell Foreign Function Interface 1.0: An Addendum to the Haskell 98
  Report.
\bibitem[Chakravarty (2014)]{language-c-inline}
  Chakravarty, M. M. (2014, September).
  Foreign inline code: systems demonstration.
  In Proceedings of the 2014 ACM SIGPLAN symposium on Haskell (pp. 119-120).
  ACM.
\bibitem[Chakravarty, Keller, Peyton Jones (2005)]{typefamilies}
  Chakravarty, M. M., Keller, G., \& Peyton Jones, S. (2005, September).
  Associated type synonyms.
  In ACM SIGPLAN Notices (Vol. 40, No. 9, pp. 241-253). ACM.
\bibitem[Claessen (1999)]{poormansconcurrencymonad}
  Claessen, K. (1999).
  A poor man's concurrency monad.
  Journal of Functional Programming, 9(03), 313-323.
\bibitem[Dijkstra, Stutterheim, Vermeulen, Swierstra (2013)]{uhc}
  Dijkstra, A., Stutterheim, J., Vermeulen, A., \& Swierstra, S. D. (2013).
  Building JavaScript applications with Haskell.
  In Implementation and Application of Functional Languages (pp. 37-52).
  Springer Berlin Heidelberg.
\bibitem[Done (2015)]{fay}
  Done, C.
  Fay programming language.
  Available from \url{http://fay-lang.org/}, 2015-05-15.
\bibitem[The 2015 ECMAScript Language Specification, Rev. 38 (2015)]{es6}
  The 2015 ECMAScript Language Specification, Rev. 38, Final Draft.
  Available from \url{http://tinyurl.com/es6draft},
  (2015-05-13).
\bibitem[Eisenberg et al (2014)]{closedtypefamilies}
  Eisenberg, R. A., Vytiniotis, D., Peyton Jones, S., \& Weirich, S. (2014).
  Closed type families with overlapping equations.
  ACM SIGPLAN Notices, 49(1), 671-683.
\bibitem[Ekblad, Claessen (2014)]{haste}
  Ekblad, A., \& Claessen, K. (2014, September).
  A seamless, client-centric programming model for type safe web applications.
  In Proceedings of the 2014 ACM SIGPLAN symposium on Haskell (pp. 79-89). ACM.
\bibitem[Epstein, Black, Peyton Jones (2011)]{cloudhaskell}
  Epstein, J., Black, A. P., \& Peyton Jones, S. (2011, September).
  Towards Haskell in the cloud.
  In ACM SIGPLAN Notices (Vol. 46, No. 12, pp. 118-129). ACM.
\bibitem[The Haste language website (2015)]{haste-lang.org}
  The Haste language website.
  Available from \url{http://haste-lang.org}, 2015-05-14.
\bibitem[Mainland (2007)]{quasiquotes}
  Mainland, G. (2007, September).
  Why it's nice to be quoted: quasiquoting for haskell.
  In Proceedings of the ACM SIGPLAN workshop on Haskell workshop (pp. 73-82).
  ACM.
\bibitem[Magalhães, Dijkstra, Jeuring, Löh (2010)]{generics}
  Magalhães, J. P., Dijkstra, A., Jeuring, J., \& Löh, A. (2010).
  A generic deriving mechanism for Haskell.
  In Haskell 2010, Baltimore, MD, USA, 30 September 2010, pages 37–48. ACM, 2010.
\bibitem[Nazarov, Mackenzie, Stegeman (2015)]{ghcjs}
  Nazarov, V., Mackenzie, H., \& Stegeman, L.
  "GHCJS Haskell to JavaScript Compiler".
  Available from \url{https://github.com/ghcjs/ghcjs}, (2014-05-14).
\bibitem[O'Sullivan (2011)]{aeson}
  O'Sullivan, B. (2011).
  The {\tt aeson} package
  Available from \url{http://hackage.haskell.org/package/aeson-0.11.1.0/}, (2016-02-29).
\bibitem[Plasmeijer, van Eekelen (2002)]{clean}
  Plasmeijer, R., \& van Eekelen, M.
  "Clean language report version 2.1".
\bibitem[Peyton Jones, Tolmach, Hoare (2001)]{rewriterules}
  Jones Peyton, S., Tolmach, A., \& Hoare, T. (2001, September).
  Playing by the rules: rewriting as a practical optimisation technique in GHC.
  In Haskell workshop (Vol. 1, pp. 203-233).
\bibitem[Reid (1994)]{stableptr}
  Reid, A. (1994, September).
  Malloc pointers and stable pointers: Improving Haskell's foreign language
  interface.
  In Glasgow Functional Programming Workshop Draft Proceedings, Ayr, Scotland.
\bibitem[Sheard, Peyton Jones (2002)]{th}
  Sheard, T., \& Jones, S. P. (2002, October).
  Template meta-programming for Haskell.
  In Proceedings of the 2002 ACM SIGPLAN workshop on Haskell (pp. 1-16). ACM.
\bibitem[Yakeley, 2014]{time}
  Yakeley, A.
  The \emph{time} package.
  Available from \url{http://hackage.haskell.org/package/time}, 2015-04-01.
\end{thebibliography}


\end{document}
