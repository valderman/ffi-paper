%-----------------------------------------------------------------------------
%
%               Template for sigplanconf LaTeX Class
%
% Name:         sigplanconf-template.tex
%
% Purpose:      A template for sigplanconf.cls, which is a LaTeX 2e class
%               file for SIGPLAN conference proceedings.
%
% Guide:        Refer to "Author's Guide to the ACM SIGPLAN Class,"
%               sigplanconf-guide.pdf
%
% Author:       Paul C. Anagnostopoulos
%               Windfall Software
%               978 371-2316
%               paul@windfall.com
%
% Created:      15 February 2005
%
%-----------------------------------------------------------------------------


\documentclass{sigplanconf}

% The following \documentclass options may be useful:

% preprint      Remove this option only once the paper is in final form.
% 10pt          To set in 10-point type instead of 9-point.
% 11pt          To set in 11-point type instead of 9-point.
% authoryear    To obtain author/year citation style instead of numeric.

\usepackage{amsmath}
\usepackage{natbib}
\usepackage{amsmath}
\usepackage{amssymb}
\usepackage{amsthm}
\usepackage[utf8]{inputenc}
\usepackage{parskip}
\usepackage{color}
\usepackage{url}
\usepackage{enumerate}  % Needed by Pandoc

\theoremstyle{definition}
\newtheorem{theorem}{Theorem}
\newtheorem{lemma}{Lemma}
\newtheorem{definition}{Definition}

\newif\ifbeamer
%\beamertrue % Uncomment when using Beamer

\newif\iflsthaskell
\lsthaskelltrue % Uncomment when using listings and Haskell

%%%%%%%%%%%%%%%%%%%%%%%%%%%%%%%%%%%%%%%%%%%%%%%%%%%%%%%%%%%%%%%%%%%%%%%%%%%%%%%%%%%%%%%%%%%%%%%%%%%%

\ifbeamer

  \setbeamertemplate{blocks}[rounded][shadow=true]
  \setbeamercolor{block body}{bg=black!17}
  \hypersetup{colorlinks,linkcolor=blue,anchorcolor=blue,citecolor=blue,filecolor=blue,urlcolor=blue}

\else

  %\usepackage[colorlinks,linkcolor=blue,anchorcolor=blue,citecolor=blue,filecolor=blue,linkcolor=blue,urlcolor=blue]{hyperref}
  \urlstyle{sf}

\fi

%%%%%%%%%%%%%%%%%%%%%%%%%%%%%%%%%%%%%%%%%%%%%%%%%%%%%%%%%%%%%%%%%%%%%%%%%%%%%%%%%%%%%%%%%%%%%%%%%%%%

\definecolor{haskell_frameColor}{rgb}{0.75,0.75,0.75}
\definecolor{haskell_codeColor}{rgb}{0.86,0.85,0.89}

\iflsthaskell
  \usepackage{float}
\usepackage{listings}
\usepackage[scaled]{beramono}

\makeatletter
\newcommand\fs@plainruled{\def\@fs@cfont{\rmfamily}\let\@fs@capt\floatc@plain%
  \def\@fs@pre{\hrule\kern2pt}%
  \def\@fs@mid{\kern2pt\hrule\vspace\abovecaptionskip\relax}%
  \def\@fs@post{}%
  \let\@fs@iftopcapt\iffalse}
\makeatother

\floatstyle{plainruled}

\newfloat{listingfloat}{tp}{lop}
\floatname{listingfloat}{\bf Figure}

\lstloadlanguages{Haskell}

\lstset{
    language=Haskell,
    keywords={where,let,in,if,then,else,case,of,do,type,data,class,instance,family
             ,newtype,deriving,import},
    otherkeywords={::},
    morecomment=[l]{--},
    literate=
        {>=}{{$\geq$}}2
        {<=}{{$\leq$}}2
%        {\\}{{$\lambda$}}1
        {\\\\}{{\char`\\\char`\\}}1
        {->}{{$\rightarrow$}}2
        {<-}{{$\leftarrow$}}2
        {=>}{{$\Rightarrow$}}2
        {<>}{{$\circ$}}1
        {>>=}{{>>=}}3
        {>>}{{>{>}}}2
        {===>}{{$\Longrightarrow$}}3
        {|->}{{$\mapsto$}}2
        {/\\}{{$\cap$}}2
        {forall}{{$\forall$}}0
        {_=()}{{$\ldots$}}3
        {OLD}{}0
        {NEW}{}0
        {BAD}{}0
        {¤}{}0
        {ZERO}{{$_{\text{0}}$}}1
        {ONE}{{$_{\text{1}}$}}1
        {TWO}{{$_{\text{2}}$}}1
        {THREE}{{$_{\text{3}}$}}1
        {FOUR}{{$_{\text{4}}$}}1
        {FIVE}{{$_{\text{5}}$}}1
        {_PLUS}{{$_+$}}1
        {_TRIANG}{{$_\triangle$}}1
        {_DIAMOND}{{$_\diamond$}}1
        {_FO}{{$_{\text{FO}}$}}2
        {_HO}{{$_{\text{HO}}$}}2
        {\\/}{{$\sqcup$}}2
        {prop_}{{$\forall$}}1
        {=id$}{{$.$}}1
        {_NS}{{$_{\text{NS}}$}}2
        {_SPEC}{{$_{\text{SPEC}}$}}3
        {_CIRC}{{$_{\text{CIRC}}$}}3
        {>>>>}{{$>$}}2
        {_CXT}{{$_{\text{CXT}}$}}3
        {PROOF}{}0
        {_V}{{$_\text{V}$}}1
        {_BV}{{$_\text{BV}$}}2
        {-->}{{$\longrightarrow$}}4
}

\ifbeamer
\else
  \lstnewenvironment{code}{\minipage{\linewidth}}{\endminipage}
    % Already set by Pandoc's default Beamer template. There doesn't seem to be a way to renew
    % listings environments. But this environment is only needed when using Pandoc with
    % `-t beamer+lhs`, so the solution is to use `-t beamer`.
\fi

\lstnewenvironment{codex}{\minipage{\linewidth}}{\endminipage}
\lstnewenvironment{ghci}{\minipage{\linewidth}\lstset{keywords={Main}}}{\endminipage}

%\lstnewenvironment{code}{\lstset{basicstyle=\fontsize{8}{10}\fontencoding{T1}\ttfamily}}{}
%\lstnewenvironment{codex}{\lstset{basicstyle=\fontsize{8}{10}\fontencoding{T1}\ttfamily}}{}
%\lstnewenvironment{ghci}{\lstset{basicstyle=\fontsize{8}{10}\fontencoding{T1}\ttfamily}\lstset{keywords={Main}}}{}


  \ifbeamer

    \lstset{
        basicstyle=\fontsize{8}{10}\fontencoding{T1}\ttfamily,
        breaklines=false,
        breakatwhitespace=false,
        xleftmargin=0.3em,
        keepspaces=true,
        % columns=flexible,
        showstringspaces=false,
        frame=single,
        rulecolor=\color{haskell_frameColor},
        backgroundcolor=\color{haskell_codeColor}
    }

  \else

    \lstset{
        basicstyle=\fontsize{8}{10}\fontencoding{T1}\ttfamily,
        breaklines=false,
        breakatwhitespace=false,
        xleftmargin=0.9em,
        keepspaces=true,
        % columns=flexible,
        showstringspaces=false
    }

  \fi
\fi


%%%%%%%%%%%%%%%%%%%%%%%%%%%%%%%%%%%%%%%%%%%%%%%%%%%%%%%%%%%%%%%%%%%%%%%%%%%%%%%%%%%%%%%%%%%%%%%%%%%%

\newcommand{\ignore}[1]{}

\newcommand{\todo}[1]{~\newline\textcolor{red}{\underline{TODO}: {\em #1}}\newline}

\lstdefinelanguage{JavaScript}{
  keywords={break, case, catch, continue, debugger, default, delete, do, else, finally, for, function, if, in, instanceof, new, return, switch, this, throw, try, typeof, var, void, while, with},
  morecomment=[l]{//},
  morecomment=[s]{/*}{*/},
  morestring=[b]',
  morestring=[b]",
  sensitive=true
}


\begin{document}
\special{papersize=8.5in,11in}
\setlength{\pdfpageheight}{\paperheight}
\setlength{\pdfpagewidth}{\paperwidth}

\exclusivelicense
\conferenceinfo{Haskell Symposium~'15}{September 3--4, 2015, Vancouver, Canada} 
\copyrightyear{2015}
\copyrightdata{bogus}
\doi{bogus}

% Uncomment one of the following two, if you are not going for the 
% traditional copyright transfer agreement.


%\permissiontopublish             % ACM gets nonexclusive license to publish
                                  % (paid open-access papers, 
                                  % short abstracts)

\titlebanner{-- Haskell Symposium '15 draft submission --}        % These are ignored unless
\preprintfooter{-- -- -- -- draft -- -- -- --}   % 'preprint' option specified.

\title{Foreign Exchange at Low, Low Rates}
\subtitle{A lightweight, platform-agnostic FFI for high level Haskell}

\authorinfo{Anton Ekblad}
           {Chalmers University of Technology}
           {antonek@chalmers.se}

\maketitle

\begin{abstract}
  We present a novel yet simple foreign function interface for Haskell dialects
  targeting high level platforms. The interface automates marshalling,
  eliminates boilerplate code, allows increased sanity checking of external
  data, and is implementable as a plain Haskell '98 library across a range of
  Haskell compilers and host environments without any modification to the
  compiler or environment.
  
  We give an implementation of this interface for the JavaScript-targeting
  Haste compiler, and show how the basic interface can be
  further optimized with minimal effort to perform on par with Haskell's
  vanilla foreign function interface, as well as extended to support automatic
  marshalling of functions and modified to support a larger range of host
  environments at the cost of a slight increase in implementation complexity.
\end{abstract}

\category{D.1.1}{Programming Techniques}{Applicative (functional) Programming}
\category{D.2.11}{Software Architectures}{Languages}
\category{D.2.12}{Interoperability}{Interface definition languages}

% general terms are not compulsory anymore, 
% you may leave them out
%\terms
%term1, term2
%
\keywords
compilers; interoperability; web

\section{Introduction}\label{sec:intro}

Interfacing with other languages is one of the more painful aspects of modern
day Haskell development. Consider figure \ref{lst:bad-ffi}, taken from the
standard libraries of GHC; a piece of code to retrieve the
current time\ \cite{time}. A relatively simple task, yet its implementation is
surprisingly complex.

\begin{listingfloat}
\begin{code}
data CTimeval = MkCTimeval CLong CLong

instance Storable CTimeval where
	sizeOf _ = (sizeOf (undefined :: CLong)) * 2
	alignment _ = alignment (undefined :: CLong)
	peek p = do
		s   <- peekElemOff (castPtr p) 0
		mus <- peekElemOff (castPtr p) 1
		return (MkCTimeval s mus)
	poke p (MkCTimeval s mus) = do
		pokeElemOff (castPtr p) 0 s
		pokeElemOff (castPtr p) 1 mus

foreign import ccall unsafe "time.h gettimeofday"
   gettimeofday :: Ptr CTimeval -> Ptr () -> IO CInt

getCTimeval :: IO CTimeval
getCTimeval = with (MkCTimeval 0 0) $ \ptval -> do
  throwErrnoIfMinus1_ "gettimeofday" $ do
    gettimeofday ptval nullPtr
  peek ptval
\end{code}
\caption{The current state of foreign imports}
\label{lst:bad-ffi}
\end{listingfloat}

This code snippet is more akin to thinly veiled C code than idiomatic, readable
Haskell; an unfortunate reality of working with the standard foreign function
interface. But is this really the best we can do?

When the world we are interfacing with uses C as its lingua franca, then the
FFI may be quite enough. However, Haskell does not only compile to
traditional C-based architectures, but to higher level environments - most
prominently JavaScript - as well. In writing this paper, we seek to liberate
Haskell code targeting JavaScript and similar high level platforms from the
shackles of the low level Foreign Function Interface traditionally used to
interface Haskell with C.

The FFI prioritizes compatibility via a lowest common denominator language: C.
To avoid making communications with our high level target platforms subject
to this unfortunate detour via an interface that fits neither Haskell nor
our target platform, we desire the following properties in a new FFI for a
hosted Haskell dialect:

\begin{itemize}
\item
  It should automatically take care of marshalling for any types where
  marshalling is defined, without extra conversions or other boilerplate code.
\item
  Users should be able to easily define their own marshalling schemes for
  arbitrary types.
\item
  It should allow importing arbitrary host language code, not just named,
  statically known functions.
\end{itemize}

Making this list a bit more concrete in the form of an example, we would like
to write high level code like that in figure \ref{lst:good-ffi}.

\begin{listingfloat}
\begin{code}
  data UTCTime = UTCTime {
      secs  :: Word,
      usecs :: Word
    } deriving Generic
  instance FromAny UTCTime

  getCurrentTime :: IO UTCTime
  getCurrentTime =
  host "() => {var ms = new Date().getTime();
              \\return {secs:  Math.trunc(ms/1000),\\
              \\        usecs: (ms % 1000)*1000};}"
\end{code}
\caption{The desired state of foreign imports}
\label{lst:good-ffi}
\end{listingfloat}

Contrasting this with the standard FFI code from figure \ref{lst:bad-ffi}:

\begin{itemize}
\item
  The low level C types are gone, replaced by a more descriptive record type,
  and so is the \lstinline!peek!ing and \lstinline!poke!ing of pointers.
\item
  The imported function arrives ``batteries included'', on equal footing with
  every other function in our program. No extra scaffolding or boilerplate
  code is necessary.
\item
  We have not only imported the \lstinline!gettimeofday! system call,
  but \emph{implemented} it as well.
  We no longer need to take the long way around with external header
  files and named, statically known functions.
\end{itemize}

\paragraph{A note on ``hosted'' Haskell}
Every programming language is in some sense hosted, where ``hosting'' refers
to the act of providing said language with an environment in which to execute:
assembly code is hosted by a physical machine and (usually) an operating
system; some languages are hosted by C, using it as a compilation target and
sharing its runtime system rather than compiling straight to machine code;
and programs written in F\# or Java are hosted by the .NET framework and the
Java Virtual Machine respectively.

For the purpose of this paper, we take a language or dialect being
\emph{hosted} to mean that it utilizes a higher level language than the
traditional operating system and C assembly setup for its compilation target
and runtime environment. In practice, this often implies JavaScript and its
associated browser-provided runtime environments, web applications currently
being very much in fashion, but applies equally well to the aforementioned
.NET and JVM platforms.

\paragraph{Our Contribution}\label{sec:contrib}
In section \ref{sec:interface}, we present a novel interface for a hosted
Haskell dialect to interface with its host language at a high level of
abstraction, and describe its implementation for the Haste compiler.
\ \cite{haste}

The interface is implementable using plain Haskell '98 with only the Foreign
Function Interface extension, and is extensible by the user in the types of
data which can be marshalled between Haskell and host language, as well as in
how those types are marshalled. It allows for context dependent sanity
checking of incoming data from the host language, improving the safety
of foreign functions. It generalizes over host languages in the sense that it
is usable with any hosted Haskell dialect where the host language supports
garbage collection, first class functions, and an \lstinline!eval!-like
construct.

In section \ref{sec:optimizations} we discuss various safety and performance
concerns about our implementation, and show how these concerns can be
alleviated by reaching outside the confines of Haskell '98.

In section \ref{sec:extensions} we show the flexibility of our design by using
it to implement marshalling of functions between the host and guest language,
and use the GHC generics extension to implement a default marshalling behavior
for any Haskell type. We also show how to lift the requirement that the host
language supports dynamic code evaluation at runtime and with a slight
modification to the Haskell compiler in use.

For clarity and consistency, we will be using the ECMAScript 6
draft specification\ \cite{es6}, from here on referred to simply as
\emph{JavaScript}, as our target platform whenever it is necessary to include
code in the host language.

\subsection{Background}\label{sec:background}
Traditionally, Haskell programs have used the Foreign Function Interface
extension\ \cite{ffi} to communicate with other languages. This works
passably well in the world of native binary programs running on bare metal,
where the C calling conventions have become the de facto standard of foreign
data interchange.
The C language has no notion of higher level data structures or
fancy data representation, making it the perfect lowest common denominator
interlingua for language to language communication:
there is no ambiguity or clash between different languages' built-in
representation of various higher level data structures, as there simply
\emph{are} no higher level data structures.

With the rise of the web browser as a major application platform, however,
a whole new breed of \emph{hosted} dialects of Haskell and other functional
programming languages have emerged, compiling not only to JavaScript and the
web browser platform, but to other high level languages as well.
New languages like Idris\ \cite{idris} offer a multitude of backends,
some of which are hosted, and compilers like \emph{Haste}\ \cite{haste} and
\emph{Ocsigen}\ \cite{ocsigen} take advantage of multiple backends to
implement \emph{program slicing}, producing programs which run concurrently
distributed over a hosted context and an unhosted, down-to-the-metal one.

For these hosted language implementations, the same properties that make
Haskell's traditional foreign function interface a good fit for language
interoperability make it undesirable as a vehicle for interfacing with
code written in the respective compiler's host language: the guest language
commonly shares a large part of its runtime with the host language and
internally uses many of the same data structures and representations, making
the forced low level representations of the vanilla foreign function interface
an unnecessary nuisance rather than a welcome common ground for data
interchange.

To further illustrate the issue, let us refer back to figure \ref{lst:bad-ffi},
which we introduced in section \ref{sec:intro}.

This code has several problems when interfacing with a high level language
rather than the pointer-dominated world of C that the FFI was designed to
cope with:

\begin{itemize}
\item
  Pervasive use of pointers not only makes the code hard to read and
  understand, but also does not map well to the underlying high level
  language.
\item
  Jumping through type-level hoops is required to appease the
  \lstinline!foreign import! declaration, which only recognizes types that
  make sense in C.
\item
  Extra scaffolding is required at all call sites to marshal data to/from
  their respective C implementations. While this is usually solved
  by hiding the foreign imports themselves and exporting wrapper functions to
  save the module's consumer from having to deal with these details, as is done
  in this example, this still requires someone to write that extra
  boilerplate code.
\item
  The FFI only lets us import named external functions, but our high level
  host language may support higher order functions - it would be nice to be able
  to take advantage of this fact.
\end{itemize}

In addition to the above points, while not exhibited by this particular
example, passing callbacks across the foreign function interface is highly
problematic as well.

When working in a hosted Haskell dialect, we ought to be able to do better
than this: bound no more by the fetters of C, hosted Haskell should be able
to interface with its host language in a less verbose and more flexible manner.

\section{A New Way}\label{sec:interface}
Our interface is based on the idea that the entirety of an interaction with the
host environment need not be an atomic operation. Rather, we attempt to break
foreign calls down into its constituent parts: marshalling arguments into the
target language, performing the actual call, and finally marshalling the
results back into Haskell. We implement these parts in Haskell itself to the
extent possible, only reaching out to the host environment when it is clear
that a task is impossible to perform within the confines of Haskell.

\subsection{Crossing the Rift}
As the main purpose of a foreign interface is to shovel data back and
forth through a rift spanning two separate programming worlds, it makes sense
to begin the design of any such interface with one central question:
what data can pass through the rift and come out on the other side still making
sense?

In our method, the class of data fulfilling this criterion is embodied in the
abstract \lstinline!HostAny! data type, inhabited by host language-native
representations of arbitrary Haskell values.
A data type is then considered to be marshallable if and only if it can be
converted to and/or from \lstinline!HostAny!.
We express this notion using two appropriately named type classes.

\begin{code}
  data Any
  type HostAny = Ptr Any

  class ToAny a where
    toAny :: a -> HostAny

  class FromAny a where
    fromAny :: HostAny -> IO a
\end{code}

We might find a \lstinline!HostAny! instance for \lstinline!ToAny!
and \lstinline!FromAny! handy. Of course, \lstinline!HostAny! already being
in its host language representation form, the instances are trivial.

\begin{code}
  instance ToAny HostAny where toAny = id
  instance FromAny HostAny where FromAny = return
\end{code}

We must then provide some means to convert values of primitive types to their
host language representation. This is a simple proposition, as this is the
forte of the vanilla foreign function interface.

\begin{code}
  foreign import ccall intToAny :: Int -> HostAny
  foreign import ccall anyToInt :: HostAny -> IO Int

  instance ToAny Int where toAny = intToAny
  instance FromAny Int where fromAny = anyToInt
  ...
\end{code}

However, if passing simple values was all we wanted to do, then there would be
no need to look any further than the vanilla foreign function interface.
We must also provide some way of combining values into more complex values,
to be able to represent lists, record types and other conveniences we take for
granted in our day to day development work.
But how should these values be combined? Depending on our host language, we
may have different primitive data structures at our disposal.

Fortunately, most host languages of interest support two basic aggregate types,
which are sufficient to represent values of any type: arrays and dictionaries.

For the purpose of brevity, we assume that we have
access to a pair of functions
\lstinline!arrToList :: FromAny a => HostAny -> [a]!
and \lstinline!listToArr:: ToAny a => [a] -> HostAny!
which are used to implement the \lstinline!FromAny! and \lstinline!ToAny!
instances respectively for lists;
they are trivial to implement either in Haskell using the vanilla foreign
function interface to gradually build a list of \lstinline!HostAny! values,
or on the host language side exploiting knowledge of the compiler's
data representation.

For dictionaries, the conversion is not as clear-cut. Depending on the data
we want to convert, the structure of our desired host language representation
of two values may well be different even when their guest language
representations are quite similar, or even identical.
Hence, we need to put the power over this decision into the hands of the user,
providing functionality to build as well as inspect user-defined dictionaries.

We will need three basic host language operations: creating a new dictionary,
associating a dictionary key with a particular value, and looking up values
from dictionary keys. From these we construct two functions to marshal compound
Haskell values to and from dictionaries: \lstinline!mkDict!
and \lstinline!getMember!, as shown in figure \ref{lst:dicts}.

\begin{listingfloat}
\begin{code}
  foreign import ccall
    newDict :: IO HostAny
  foreign import ccall
    set :: HostAny -> HostString -> HostAny -> IO ()
  foreign import ccall
    get :: HostAny -> HostString -> IO HostAny
  foreign import ccall
    has :: HostAny -> HostString -> IO Bool
  
  mkDict :: [(String, HostAny)] -> HostAny
  mkDict xs = unsafePerformIO $ do
    d <- newDict
    mapM_ (\k v -> set d (toHostString k) v) xs
    return d
    
  getMember :: FromAny a => HostAny -> String -> IO (Maybe a)
  getMember dict key = do
    let key' = toHostString key
    has_key <- dict `has` key'
    if has_key
      then get dict key' >>= fmap Just . fromAny
      else return Nothing
\end{code}
\caption{Reading and writing dictionaries}
\label{lst:dicts}
\end{listingfloat}

This gives us the power to represent any composite or primitive data
type. To give an intuition of the interface's use, figure
\ref{lst:tuplemaybeinstances} gives possible \lstinline!ToAny! and
\lstinline!FromAny! instances for tuples and the \lstinline!Maybe! type.
Note how \lstinline!Maybe a! is represented in the host language as a
dictionary which is either {\tt \{"tag": "nothing"\}} or
{\tt \{"tag": "something", "data": a\}}.

It is also worth noting that the implementation of \lstinline!getMember! is
the reason for \lstinline!fromAny! returning a value in the \lstinline!IO!
monad: foreign data structures are rarely, if ever, guaranteed to be immutable
and looking up a key in a dictionary is effectively dereferencing a reference,
so we must perform any such lookups at a well defined point in time,
lest we run the risk of the value being changed in between the application
of our marshalling function and the evaluation of the resulting thunk.

\begin{listingfloat}
\begin{code}
  instance (ToAny a, ToAny b) => ToAny (a, b) where
    toAny (a, b) = toAny [toAny a, toAny b]

  instance (FromAny a, FromAny b) => FromAny (a, b) where
    fromAny x = do
      [a, b] <- fromAny x
      a' <- fromAny a
      b' <- fromAny b
      return (a', b')

  instance ToAny a => ToAny (Maybe a) where
    toAny (Just x) =
      mkDict [("tag",  toAny "something"),
              ("data", toAny x)]
    toAny Nothing =
      mkDict [("tag",  toAny "nothing")]

  instance FromAny a => FromAny (Maybe a) where
    fromAny x = do
      tag <- getMember x "tag"
      case tag of
        Just "something" -> getMember x "data"
        Just _           -> return Nothing
        Nothing          -> error "Not Maybe!"
\end{code}
\caption{Possible instances for tuples and \lstinline!Maybe!}
\label{lst:tuplemaybeinstances}
\end{listingfloat}

With this, we are ready to tackle the actual importing of functions.

\subsection{Importing functions}\label{sec:import}

Making an effort to keep our interface as simple as possible, we would very
much like to be able to import functions using a single function.
However, imported functions should also be usable without any extra
scaffolding; what type should such a function have?
Since we want to be able to import functions of just about any type, it clearly
needs to be polymorphic. Said functions live in the effectful world of our
host language, so the imported functions should live in the \lstinline!IO!
monad in the interest of safety. We want to be able to import functions of any
arity, so it also needs to be variadic. Finally, we also need to specify the
host code to import somehow.

After some contemplation, we come up with the following type, with instances of
the \lstinline!Import! class fulfilling the above criteria: variadic functions
with polymorphic arguments and return values in the \lstinline!IO! monad.

\begin{code}
  class Import f

  host :: Import f => String -> f
\end{code}

\begin{listingfloat}
\begin{code}
  logError :: String -> IO ()
  logError = host "e => {console.error('Oops: ' + e);}"
\end{code}
\caption{Example of an imported function}
\label{lst:importexample}
\end{listingfloat}

This will let us write import declarations like in figure
\ref{lst:importexample}, creating proper Haskell functions from strings of
host language code.

The types of our imported functions need to differ depending on the arity of
the imported host language code. This necessitates \lstinline!host! returning
some variadic function. Fortunately, there is a well known trick to accomplish
this, described in\ \cite{printf}, which uses an inductive class instance to
successively build up a list of arguments over repeated function applications,
and a base case instance to perform some computation over said arguments after
the function in question has been fully applied.
In the case of the \lstinline!host! function, that computation would be
applying a foreign function to said list of arguments.

This suggests the following class definition:

\begin{code}
  type HostFun = HostAny

  class Import f where
    imp0rt :: HostFun -> [HostAny] -> a
\end{code}

For our purposes, the base case is a nullary computation in the IO monad:

\begin{code}
  foreign import ccall
    apply :: HostFun -> HostAny -> IO HostAny

  instance FromAny a => Import (IO a) where
    imp0rt f args =
      apply f (toAny args) >>= fromAny
\end{code}

Note the use of a foreign import in our base case. As the application of a
foreign function to a foreign list of foreign arguments is clearly, well, a
\emph{foreign} matter, we must call out to the host language for this final
step.
The list of arguments needs to be converted from \lstinline![HostAny]! to
\lstinline!HostAny! in order to squeeze it through the vanilla foreign
function interface, and the value we get back marshalled back into a proper
Haskell value.

The inductive case is not much more complex: we only need to marshal a single
argument and recurse.

\begin{code}
  instance (ToAny a, Import b) =>
           Import (a -> b) where
    imp0rt f args =
      \arg -> imp0rt f (toAny arg : args)
\end{code}

With this, we have all the building blocks required to implement the
\lstinline!host! function. With all the hard work already done, the
implementation is simple. For the sake of brevity, we assume the
existence of a host language specific \lstinline!HostString! type, which may
be passed as an argument over the vanilla foreign function interface, and a
function \lstinline!toHostString :: String -> HostString!.

\begin{code}
  foreign import ccall
    eval :: HostString -> HostFun

  host :: Import f => HostString -> f
  host s = imp0rt f []
    where
      f = eval (toHostString s)
\end{code}

The foreign \lstinline!eval! import brings in the host language's evaluation
construct. Recall that one requirement of our method is the existence of such
a construct, to convert arbitrary strings of host language code into functions
or other objects.

\section{Optimizing for safety and performance}
\label{sec:optimizations}
While the implementation described up until this point is more or less
feature complete, its non-functional properties can be improved quite a bit
if we allow ourselves to stray from the tried and true, but slightly
conservative, path of pure Haskell '98.

Aside from implementation specific tricks - exploiting knowledge about a
particular compiler's data representation to optimize marshalling, or even
completely unroll and eliminate some of the basic interface's primitive
operations, for instance - there are several general optimizations we can
apply to significantly enhance the performance and safety of our interface.

\subsection{Eliminating argument passing overheads}
\label{sec:specialization}
The performance-minded reader may notice something troubling about the
implementation of \lstinline!imp0rt!: the construction of an intermediate list
of arguments. Constructing this intermediate list only to convert it into a
host language suitable representation which is promptly deconstructed as soon
as it reaches the imported function takes a lot of work. Even worse, this work
does not provide any benefit for the task to be performed: applying a foreign
function.

By the power of \emph{rewrite rules}\ \cite{rewriterules}, we can
eliminate this pointless work in most cases by specializing the
\lstinline!host! function's base case instance for different numbers of
arguments.
In addition to the general \lstinline!apply! function we define a series of
\lstinline!apply0!, \lstinline!apply1!, etc. functions, one for each arity
we want to optimize function application for.
The actual specialization is then a matter of
rewriting \lstinline!host! calls to use the appropriate application function.

Figure \ref{lst:specialization} gives a new implementation of the base case
of the \lstinline!Import! class which includes this optimization, replacing
the one given in \ref{sec:interface}.

\begin{listingfloat}
\begin{code}
  {-# NOINLINE [0] host' #-}
  host' :: FromAny a => HostFun -> [HostAny] -> IO a
  host' f args = apply f (toAny args) >>= fromAny

  instance FromAny a => Import (IO a) where
    host = host'
  
  foreign import ccall
    apply0 :: HostFun -> IO HostAny
  foreign import ccall
    apply1 :: HostFun -> HostAny -> IO HostAny
  foreign import ccall
    apply2 :: HostFun -> HostAny -> HostAny -> IO HostAny
  ...
  
  {-# RULES
    "apply0" [1] forall f. host' f [] =
                 apply0 f >>= fromAny
    "apply1" [1] forall f a. host' f [a] =
                 apply1 f a >>= fromAny
    "apply2" [1] forall f a b. host' f [b,a] =
                 apply2 f a b >>= fromAny
    ...
   #- }
\end{code}
\caption{Specializing the \lstinline!host! base case}
\label{lst:specialization}
\end{listingfloat}

\subsection{Preventing code injection}
\label{sec:safe_host}
Meanwhile, the \emph{safety-conscious} reader may instead be bristling at the
thought of executing code contained in something as egregiously untyped and
untrustworthy as a common string.
Indeed, by allowing the conversion of arbitrary strings
into functions, we're setting ourselves up for cross-site scripting attacks
and other similar code injection attacks!

While this is indeed true in theory, in practice, accidentally passing a
user-supplied string to the \lstinline!host! function, which in normal use
ought to occur almost exclusively on the top level of a module, is a quite
unlikely proposition. Even so, it could be argued that if it is possible to use
an interface for evil, its users almost certainly will at some point.

Fortunately, the recent 7.10 release of the GHC compiler gives us the means to
eliminate this potential pitfall. The \emph{StaticPointers} extension, its
first incarnation described in\ \cite{cloudhaskell}, introduces the
\lstinline!static! keyword, which is used to create values of type
\lstinline!StaticPtr! from closed expressions. Attempting to turn any
expression which is not known at compile time into a \lstinline!StaticPtr!
yields a compiler error.

Implementing a \lstinline!safe_host! function which can not be used to execute
user-provided code becomes quite easy using this extension and the basic
\lstinline!host! function described in section \ref{sec:interface},
at the cost of slightly more inconvenient import syntax:

\begin{code}
  safe_host :: Import f => StaticPtr HostString -> f
  safe_host = host . deRefStaticPtr

  safe_hello :: IO ()
  safe_hello = safe_host $
    static "() => alert('Hello, world!')"
\end{code}

Note that unlike the basic interface, the \lstinline!safe_host!
function is currently \emph{not} included in our reference implementation,
as the Haste compiler it builds on does not yet support this language
extension.

\subsection{Eliminating \lstinline!eval!}
\label{sec:inline-eval}
Relying on \lstinline!eval! to produce our functions allows us to
implement our interface in pure Haskell '98 without modifying the Haskell
compiler in question, making the interface easy to understand, implement and
maintain. However, there reasons why it may be in the implementor's best
interest to forgo a small bit of that simplicity.

The actual call to \lstinline!eval! does not meaningfully impact
performance: it is generally only called once per import,
the resulting function object cached thanks to lazy evaluation.
\footnote{The main reason for \lstinline!eval! getting called more than once
being unwise inlining directives from the user.}
However, its dynamic nature \emph{does}
carry a significant risk of interfering with the ability of the host language's
compiler and runtime to analyze and optimize the resulting code.
As discussed in section \ref{sec:performance}, this effect is very much in
evidence when targeting the widely used V8 JavaScript engine.

In the case of JavaScript, it is quite common to run programs through a
\emph{minifier} - a static optimizer with focus on code size - before
deployment. Not only do such optimizers suffer the same analytical
difficulties as the language runtime itself, but due to the heavy use of
renaming often employed by minifiers to reduce code size, special care needs
to be taken when writing code that is not visible as such to the minifier:
code which is externally imported or, in our case, locked away inside a string
for later evaluation.

Noting that virtually every sane use of our interface evaluates a \emph{static}
string, a solution presents itself: whenever the \lstinline!eval! function
is applied to a statically known string, instead of generating a function call,
the compiler splices the contents of the string verbatim into the output code
instead.

This solution has the advantage of eliminating the code analysis obstacle
provided by \lstinline!eval! for the case when our imported code is statically
known (which, as we noted before, is a basic sanity property of foreign
imports), while preserving our library's simplicity of implementation.
However, it also has the \emph{dis}advantage of requiring modifications to the
compiler in use, however slight, which increases the interface's overall
complexity of implementation.

\section{Putting our interface to use}
\label{sec:extensions}
While the interface described in sections \ref{sec:interface} and
\ref{sec:optimizations} represents a clear raising of the abstraction layer
over the vanilla foreign function interface, it is still lacking some
desirable high level functionality: marshalling of arbitrary functions and
generic data types.

In this section we demonstrate the flexibility of our interface by implementing
this functionality on top of our basic interface.

\subsection{Dynamic function marshalling}
\label{sec:marshalling-functions}
\paragraph{Dynamic imports}
One appealing characteristic of our interface is that it makes the marshalling
of functions between Haskell and the host language easy. In the case of
passing host functions into Haskell, the \lstinline!imp0rt! function used to
implement \lstinline!host! has already done the heavy lifting for us.
Only adding an appropriate \lstinline!FromAny! instance remains.

Due to the polymorphic nature of functions, however, we must resort to using
some language extensions to get the type checker to accept our instance:
overlapping instances, flexible instances, and undecidable instances.
Essentially, the loosened restrictions on type class instances allows an
\lstinline!Import! instance to act as a synonym for \lstinline!FromAny!,
allowing host language functions to return functions of any type admissible
as an import type by way of the \lstinline!host! function.

\begin{code}
  instance Import a => FromAny a where
    fromAny f = return (imp0rt f [])
\end{code}

\paragraph{Passing functions to foreign code}
Passing functions the other way, out of Haskell and into our host language,
requires slightly more work. While we already had all the pieces of the
dynamic import puzzle at our disposal through our earlier implementation of
\lstinline!host!, exports require one more tool in our toolbox: a way to turn
a Haskell function into a native host language function.

Much like the \lstinline!apply! function used in the implementation of
\lstinline!host!, the implementation of such a function is specific to the
host language in question. Moreover, as we are dealing with whatever format
our chosen compiler has opted to represent functions by, this function is also
dependent on the compiler.

In order to implement this function, we assume the existence of another
function \lstinline!hfsun_to_host!, to convert a Haskell function $f$ from $n$
\lstinline!HostAny! arguments to a \lstinline!HostAny! return value $r$ in the
IO monad into a host language function which, when applied to $n$ host
language arguments, calls $f$ with those same arguments and returns the $r$
returned by $f$.

\begin{code}
  foreign import ccall
    hsfun_to_host :: ??? -> HostFun
\end{code}

But how can we make this function type check?
As we are bound to the types the vanilla foreign function interface lets us
marshal, we have no way of applying this function to a variadic Haskell
function over \lstinline!HostAny!s.

We know that, operationally, \lstinline!hsfun_to_host! expects a Haskell
function as its input, but the types do not agree; we must somehow find a way
to pass arbitrary data unchanged to our host language.
Fortunately, standard Haskell provides us with a way to do exactly what we want:
\lstinline!StablePointer!s.\ \cite{stableptr} Note that, depending on the
Haskell compiler in use, this use of stable pointers may introduce a space
leak. This is discussed further in section \ref{sec:host-gc}, and an
alternative solution is presented.

\begin{code}
  import Foreign.StablePtr
  import System.IO.Unsafe

  foreign import ccall
    hsfun_to_host' :: StablePtr a -> HostFun

  hsfun_to_host :: Exportable f => f -> IO HostFun
  hsfun_to_host f =
    hsfun_to_host' `fmap` newStablePtr (mkHostFun f)
\end{code}

Just being able to pass functions to the host language is not enough: we still
need to marshal return values from Haskell to our host language, and incoming
arguments in the opposite direction. Essentially, we want to map
\lstinline!fromAny! over all input arguments to a function, and
\lstinline!toAny! over its return values.
While superficially similar to the implementation of the \lstinline!Import!
class in section \ref{sec:import}, this task is surprisingly tricky:
where \lstinline!Import! modifies an arbitrary number of arguments and
performs some action with respect to a monomorphic value - the
\lstinline!HostFun! representation of a host language function - we now need
to do the same to a variadic function.

\paragraph{Modifying variadic functions using type families}
A straightforward application of the \lstinline!Import! method is not flexible
enough to tackle this problem. Instead, we bring in yet another language
extension, closed type families\ \cite{closedtypefamilies}, to lend us the type
level flexibility we need. We begin by defining the \lstinline!Exportable!
type class first encountered in the type signature of \lstinline!hsfun_to_host!,
and a closed type family describing the type level behavior of our function
marshalling.

\begin{code}
  type family Host a where
    type Host (a -> b) = a -> Host b
    type Host (IO a)   = IO HostAny

  class Exportable f where
    mkHostFun :: f -> Host f
\end{code}

This is relatively straightforward. Inspecting the \lstinline!Host! type
family, we see that applying \lstinline!mkHostFun! to any eligible function
must result in a corresponding function of the same arity, but with its
arguments and return values replaced by \lstinline!HostAny!.

Giving the relevant \lstinline!Exportable! instances is now mostly a matter of
making the types match up, and concocting a \lstinline!ToAny! instance is only
a matter of composing our building blocks together.

\begin{code}
  instance ToAny a => Exportable (IO a) where
    mkHostFun = fmap toAny

  instance (FromAny a, Exportable b) =>
           Exportable (a -> b) where
    mkHostFun f =
      mkHostFun . f . unsafePerformIO . fromAny
  
  instance Exportable f => ToAny f where
    {-# NOINLINE toAny #-}
    toAny = unsafePerformIO . hsfun_to_host
\end{code}

The one interesting instance here is that of the inductive case, where we use
\lstinline!fromAny! in conjunction with \lstinline!unsafePerformIO! to marshal
a single function argument. While using \lstinline!fromAny! outside the
\lstinline!IO! monad is unsafe in the general case as explained in section
\ref{sec:interface}, this particular instance is completely safe, provided that
\lstinline!mkHostFun! is \emph{not} exported to the user, but only used to
implement the \lstinline!ToAny! instance for functions.

\lstinline!fromAny! will only be called unsafely when a function is marshalled
into a \lstinline!HostAny! value and \emph{then} applied. There are two cases
when this can happen: either the marshalled function is called from the host
language, or it is marshalled back into Haskell and then applied.
In the former case, the time of the call is trivially well defined assuming
that our target language is not lazy by default, which is rather unlikely.
In the latter case, the time of the call is still well defined, as our
interface only admits importing functions in the \lstinline!IO! monad.

Slightly more troubling is the use of \lstinline!unsafePerformIO! in
conjunction with \lstinline!hsfun_to_host!.
According to\ \cite{stableptr}, the creation of stable pointers residing in
the \lstinline!IO! monad - the reason for \lstinline!hsfun_to_host! residing
there as well - is to avoid accidentally duplicating the allocation of the
stable pointer, something we can avoid by telling the compiler to never
inline the function, ever.

It is also worth pointing out that the concern over duplicating this allocation
is only valid where this implementation espouses the aforementioned space leak
problem, in which case the alternative implementation given in section
\ref{sec:host-gc} should be preferred anyway.

\paragraph{Marshalling pure functions}
The above implementation only allows us to pass functions in the \lstinline!IO!
monad to foreign code, but we would also like to support passing pure
functions. There are two main obstacles to this:

\begin{itemize}
\item
  The \lstinline!hsfun_to_host'! function expects a function in the
  \lstinline!IO! monad.
\item
  Instantiating \lstinline!Exportable! for any type \lstinline!ToAny t => t!
  would accidentally add a \lstinline!ToAny! instance for
  \emph{any type at all}.
  Even worse, this instance is completely bogus for most types, always treating
  the argument to its \lstinline!toAny! implementation as a function to be
  converted into a host language function!
\end{itemize}

We sidestep the first problem by assuming that \lstinline!hsfun_to_host'!
can determine dynamically whether a function is pure or wrapped in the
\lstinline!IO! monad, and take action accordingly.
Another, slightly more verbose, possibility would be to alter the
implementation of our marshalling code to use either
\lstinline!hsfun_to_host'! or a function performing the same conversion
on pure functions, depending on the type of function being marshalled.

Looking closer at the problematic \lstinline!ToAny! instance, we find that
the \lstinline!Exportable t => ToAny t! instance provides
\lstinline!ToAny! for any \lstinline!Exportable! type, and the
\lstinline!ToAny t => Exportable t! instance provides \lstinline!Exportable! in
return, creating a loop which creates instances for both type classes
matching any type.

The \lstinline!ToAny t => Exportable t! instance is necessary
for our type level recursion to work out when marshalling pure functions,
but we can prevent this instance from leaking to \lstinline!ToAny! where it
would be unreasonably broad by replacing our \lstinline!ToAny! function
instance with two slightly more specific ones.

Figure \ref{lst:export-final} gives our final implementation of dynamic
function exports.
Looking at this code we also see why the use of closed type families are
necessary: the open type families originally introduced by Chakravarty et al
in\ \cite{typefamilies} do not admit the overlapping type equations required
to make pure functions an instance of \lstinline!Exportable!.

\begin{listingfloat}
\begin{code}
  import Foreign.StablePtr
  import System.IO.Unsafe

  foreign import ccall
    hsfun_to_host' :: StablePtr a -> HostFun

  hsfun_to_host :: Exportable f => f -> IO HostFun
  hsfun_to_host f =
    hsfun_to_host' `fmap` newStablePtr (mkHostFun f)

  type family Host a where
    type Host (a -> b) = a -> Host b
    type Host (IO a)   = IO HostAny
    type Host a        = HostAny

  instance (ToAny a, Host a ~ HostAny) => HostFun a where
    mkHostFun = toAny

  instance (FromAny a, Exportable b) => ToAny (a -> b) where
    {-# NOINLINE toAny #-}
    toAny = unsafePerformIO . hsfun_to_host

  instance ToAny a => ToAny (IO a) where
    {-# NOINLINE toAny #-}
    toAny = unsafePerformIO . hsfun_to_host
\end{code}
\caption{Dynamic function exports implemented on top of our interface}
\label{lst:export-final}
\end{listingfloat}

\subsection{Static function exports}
\label{sec:exports}
Very rarely are users prepared to abandon person-decades of legacy code;
to reach these users, the ability to expose Haskell functionality to the host
language is important. Alas, being implemented as a library, our interface is
not capable of \lstinline!foreign export! declarations. We can, however,
implement a substitute on top of it.

Rather than a writing a library which when compiled produces a shared library
for consumption by a linker, we give the user access to a function
\lstinline!export! which when executed stores an exported function in a
known location, where foreign language code can then access it.
While this may seem like a silly workaround, this is how JavaScript programs
commonly ``link against'' third party libraries.

Using the function marshalling implemented in section
\ref{sec:marshalling-functions}, implementing \lstinline!export! becomes a
mere matter of passing a function to the host language, which then
arranges for the function to be available in a known, appropriate location.

\begin{code}
  export :: Exportable f => String -> f -> IO ()
  export =
    host "(name, f) => {window['haskell'][name] = f;}"
\end{code}

\subsection{Generic marshalling}
\label{sec:default-instances}
Returning to our motivating example with figure \ref{lst:good-ffi}, we note
a conspicuous absence: the \lstinline!UTCTime! instance of \lstinline!FromAny!
is not defined, yet it is still used by the \lstinline!host! function in the
definition of \lstinline!getCurrentTime!. Although the instance can be defined
in a single line of code, it would still be nice if we could avoid the tedium
of writing that one line altogether. Thanks to generic programming and default
type class instances, we can.

Our implementation of generic marshalling uses GHC generics\ \cite{generics},
and associated language extensions - most notably type operators and scoped
type variables - making it specific to GHC-based compilers such as Haste and
GHCJS\ \cite{ghcjs}. GHC generics allows us to traverse values of any type
as though the type was uniformly defined as a tree of sums, products,
constants and metadata, such as record selectors or constructor names.

For the sake of brevity, and as the actual syntax of GHC generics is
relatively nondescriptive due to its generality, we only give the basic method
of our implementation is this paper, and only consider the case of marshalling
Haskell values into their host language counterparts.
Marshalling in the other direction uses the same basic method,
and the complete implementation is available from \cite{haste-lang.org}.

We begin by defining a the data type to represent a host language value while
it is being constructed. A value can be either a singleton, a list of
values or a dictionary.

\begin{code}
  data Value
    = One HostAny
    | List [HostAny]
    | Dict [(HostString, HostAny)]
\end{code}

We then informally define the behavior of our generic marshalling function
\lstinline!gToAny :: Rep a -> Value! as follows, where \lstinline!Rep! is
a type provided by \lstinline!GHC.Generics! to enable generic traversal
of its type argument.

\begin{itemize}
\item
  When we reach a \emph{constructor argument} \lstinline!x! of a type
  \lstinline!t! with a \lstinline!ToAny! instance, we use that instance to
  marshal \lstinline!x! and return it as a single value:
  \lstinline!One (toAny x)!.
\item
  When we reach a \emph{record selector} metadata node with a selector
  name \lstinline!n! and a child node \lstinline!c!, we recursively marshal
  \lstinline!c! and return it paired with its selector name:
  \lstinline!Tree [(n, toAny x)]!.
\item
  When we reach a \emph{constructor} metadata node with a constructor name
  name \lstinline!n! and a child node \lstinline!c!, we recursively marshal
  \lstinline!c! and call the resulting value \lstinline!c'!.
  \begin{itemize}
  \item
    \lstinline!c'! is a \emph{dictionary}, we add an entry to it to mark the
    value's constructor name and return the resulting dictionary:
    \lstinline!Tree (("tag", toAny n) : c')!.
  \item
    \lstinline!c'! is an \emph{empty list}, we simply return the constructor
    name:
    \lstinline!One (toAny n)!.
  \item
    \lstinline!c'! is a nonempty \emph{list} or a \emph{single item},
    we return a new dictionary consisting of the constructor tag and the
    \lstinline!HostAny! encoding of \lstinline!c'!:
    \lstinline!Tree [("tag", toAny n), ("data", toAny c')]!.
  \end{itemize}
\item
  When we reach a \emph{product} node with child nodes \lstinline!c1! and
  \lstinline!c2!, signifying the union of two or more constructor arguments,
  we recursively marshal \lstinline!c1! and \lstinline!c2! into
  \lstinline!c1'! and \lstinline!c2'! respectively. We then \emph{merge}
  \lstinline!c1'! and \lstinline!c2'! and return the result:

  \begin{code}
    merge c1' c2'
      where
        merge (One a) (One b)   = List [a, b]
        merge (List a) (One b)  = List (a ++ [b])
        merge (One a) (List b)  = List (a : b)
        merge (List a) (List b) = List (a ++ b)
        merge (Tree a) (Tree b) = Tree (a ++ b)
  \end{code}

  Note that the case where a tree is merged with a non-tree is undefined.
  Trees arise \emph{only} from a use of record selectors. Haskell only
  allows data constructors where either \emph{all} arguments have selectors,
  or \emph{none} has, meaning that trees and non-trees will never appear in
  the same product node.
\item
  When we reach a \emph{sum} node, signifying one of several data constructors
  of a type, we will either have a \emph{left} child or a \emph{right} child.
  We simply recurse down through whichever child node we have and return the
  result.
\end{itemize}

Using this implementation, all that remains is to add a default instance to
the \lstinline!ToAny! class.

\begin{code}
  class ToAny a where
    toAny :: a -> HostAny
    default toAny :: (GToAny (Rep a), Generic a)
                  => a -> HostAny
    toAny x =
      case gToAny (from x) of
        One x   -> x
        List xs -> toAny xs
        Tree d  -> mkDict d
\end{code}

This allows any type with a \lstinline!Generic! instance to automatically
instantiate \lstinline!ToAny!, enabling us to write code like that in figure
\ref{lst:generic-toany}.

\begin{listingfloat}
  \begin{code}
    > data User = User
    >   { name  :: String,
    >   , email :: String,
    >   , age   :: Int}
    >   deriving Generic
    > instance ToAny User

    > printAny :: ToAny a => a -> IO ()
    > printAny = host "x => {console.log(x.toString());}"
    
    > printAny $ User
    >    { name  = "John Doe"
    >    , email = "johndoe@example.com"
    >    , age   = 32}

    {'name'  : 'John Doe',
     'email' : 'johndoe@example.com',
     'age'   : 32}
  \end{code}
\caption{Generic marshalling in action}
\label{lst:generic-toany}
\end{listingfloat}

\section{Performance}
\label{sec:performance}
It must be said, that performance is not a significant motivator for this work.
High level features oftentimes come at the price of decreased performance,
and calling back and forth between Haskell and the host environment is usually
not cheap under the best circumstances.
Even so, having a picture of how a particular tool performs vis a vis its
alternative in the programmer's toolbox is a very useful, even essential,
thing.

We have measured the performance of an implementation of our interface for the
Haste compiler, applying the optimizations described in section
\ref{sec:optimizations}, against the Haste compiler's implementation of the
vanilla FFI. To establish a baseline for basic marshalling rather than
comparing apples to oranges by attempting to include more complex types
into our benchmarks, types which may not even be properly marshallable using
the vanilla FFI, we have chosen to only benchmark the marshalling of
\lstinline!Double!s, which map directly to JavaScript's native
\lstinline!Number! type.

To get a picture of how our library fares in different scenarios,
two microbenchmarks were devised: one which repeatedly applies a foreign
function, which performs no computation to make differences in
calling overhead stand out as much as possible, in a strict, tight, tail
recursive loop, and one which does the same in a higher level loop,
constructed from nested calls to \lstinline!mapM! over a list.
In order to separately
test the overheads of marshalling outbound and inbound data, the benchmarks
were created using two variants each: one which ignores the value returned
by the foreign function, and one which adds up the results.
These benchmarks were then run and timed
using the node.js JavaScript interpreter, with the foreign function imported
through our library and the vanilla FFI respectively.

While this may not be the most rigorous of performance evaluations,
the results are repeatable, and the methodology is enough for our purposes:
getting a rough picture of how much speed we are giving up for a more
convenient interface.
The results for each benchmark are given in table \ref{tbl:performance} as
the ratio of the run time for our library over the run time for the vanilla
FFI.

\begin{table}
  \renewcommand{\arraystretch}{1.5}
  \begin{center}
    \begin{tabular}{|r|c|l|}
      \hline
                       & Tight loop & Loose loop \\
      \hline
      Without incoming & 0.85       & 1.33 \\
      \hline
      With incoming    & 14.52      & 1.76 \\
      \hline
    \end{tabular}
  \end{center}
  \caption{Performance numbers, in relation to vanilla FFI}
  \label{tbl:performance}
\end{table}

\paragraph{Tight loop, no inbound data}
Looking at the performance numbers, our library performs surprisingly well in
a highly optimized loop, even edging out the vanilla FFI by a slight margin.
The reason for this discrepancy is not entirely clear. The only major difference
generated in the JavaScript code generated between the two is the creation and
evaluation of one extra thunk, as well as the first iteration of the loop
being unrolled, for our interface. This appears to be enough of a change that
the JavaScript interpreter is able to optimize that program slightly better
than its vanilla FFI equivalent.

\paragraph{Loose loop, no inbound data}
Putting our library into a higher level context, this small and highly
debatable performance advantage disappears. Due to our interface being
implemented mainly as a pure Haskell library, the compiler is noticeably worse
at figuring out the strictness properties of the program compiled using our
library than with the program using the vanilla FFI, leading to some
unnecessary creation and evaluation of thunks.

\paragraph{Tight loop, with inbound data}
Moving on to the benchmarks where we actually marshal incoming data, we see a
\emph{huge} performance hit on inbound data. It turns out, that this is due to
the increased safety of our interface. The vanilla foreign function
interface happily passes whatever value it gets back into Haskell as a double,
potentially wreaking type-unsafe havoc when it turns out that the ``double''
returned was actually an array of strings containing someone's shopping list.

In contrast, our library ensures that what is claimed at the type level - that
the value coming from the outside world is a \lstinline!Double! - is actually
true, by passing the marshalled value to the JavaScript built-in function
\lstinline!Number!. What this benchmark tells us is that checking if something
is a number or not is roughly 14 times slower than just copying the same value.
By replacing our safe marshalling function with one that mimics the behavior
of the vanilla FFI, we see the same performance and generated code as in the
\emph{tight loop, no inbound data} case.

\paragraph{Loose loop, no inbound data}
Moving to a less optimized loop, the percentage of the execution time spend
marshalling and calling foreign functions drops drastically, as we can see by
the performance hit of our library being much smaller than for the tight,
optimized loop. Analyzing the generated code, this is again due to the extra
safety net, with an additional slight performance disadvantage from the return
value being wrapped in a thunk, despite obviously (to a human) being strict
coming from the host language; the malus to strictness analysis incurred by
our library in action again.

\paragraph{Performance verdict: acceptable}
Judging by these numbers, disregarding the additional cost incurred by the
additional safety measures, the performance of our library is quite acceptable,
ranging from slightly faster than the vanilla FFI to at most twice as slow.
Considering that calls between Haskell and the host environment are usually
relatively few and far between, this performance difference should be
negligible. For code which has no choice but to make a large number of calls to
low level host language functions in performance critical loops,
using the vanilla FFI alongside our interface remains an option,
allowing the user to have the FFI cookie and eat it at the same time.

While the additional performance cost incurred from the additional safety may
be unwanted when the user is completely sure that a particular function
will never, ever return anything but an actual \lstinline!Number!, said user
is in free to write an \lstinline!UnsafeDouble! wrapper with an appropriately
fast and loose \lstinline!FromAny! instance to sidestep the (hopefully)
unnecessary safety measures.

\section{Limitations}\label{sec:discussion}
While two of the tree main limitations our interface places on its host
language - the presence of a dynamic code evaluation construct and support for
first class functions - have hopefully been adequately explained, and their
severity slightly alleviated, in sections \ref{sec:interface} and
\ref{sec:inline-eval}, there are still several design choices and lingering
limitations that may need further justification.

\subsection{fromAny type level expressiveness}
The \lstinline!fromAny! function used to implement marshalling in section
\ref{sec:interface} is by definition not total. As its purpose is to
convert values of god-knows-what host language type into properly typed Haskell
values, from the simplest atomic values to the most complex data structures,
the possibility for failure is apparent. Why, then, does its type not admit
the possibility of failure, for instance by wrapping the converted value in a
\lstinline!Maybe! or \lstinline!Either!?

Recall that \lstinline!fromAny! will almost always be called when
automatically converting arguments to and return values from callbacks and
imported foreign functions respectively. In this context, even if a conversion
were to fail with a \lstinline!Left "Bad conversion"! error, there is no way
for this error value to ever reach the user. The only sensible action for the
foreign call to take when encountering an error value would be to throw an
exception, informing the user ``out of band'' rather than by somehow threading
an error value to the entire call.

It is then simpler, not to mention more performant, to trust that the foreign
code in question is well behaved - something we must do anyway since we
automatically rely on its correctness by allowing our program to depend upon
it - and throw that exception immediately on conversion failure rather than
taking a detour via error values, should this trust prove to be misplaced.

\subsection{Limitation to garbage collected host languages}
\label{sec:host-gc}
The observant reader may notice that up until this point, we have completely
ignored something which very much concerns traditional foreign function
interfaces: ownership and eventual deallocation of memory.

Our high level interface depends quite heavily on its target language being
garbage collected, as having to manually manage memory introduces significant
boilerplate code and complexity - the very things this interface aims to avoid.
As target platforms \emph{with} garbage collections having to deal with low
level details such as memory management is the core motivation for this work,
rectifying this ``problem'' does not fall within the scope of this paper.

Even so, memory management does rear its ugly head in section
\ref{sec:marshalling-functions}, where stable pointers are used to pass data
unchanged from Haskell into our host language, and promptly ignored - note
the complete absence of calls to \lstinline!freeStablePtr!. Implementing our
interface for the Haste compiler, this is not an issue: Haste makes full use
of JavaScript's garbage collection capabilities to turn stable pointers into
fully garbage collected aliases of the objects pointed to.
It is, however, quite conceivable for an implementation to perform some manual
housekeeping of stable pointers even in a garbage collected language, in which
case this use of our interface will cause a space leak as nobody is keeping
track of all the stable pointers we create.

As the stable pointers in question are never dereferenced or otherwise
used within Haskell, this hypothetical space leak can be eliminated by
replacing stable pointers with a slight bit of unsafe, implementation-specific
magic.

\begin{code}
  import Unsafe.Coerce
  import Foreign.StablePtr hiding (newStablePtr)

  data FakeStablePtr a
  fakeStablePtr :: a -> FakeStablePtr a

  newStablePtr :: a -> StablePtr
  newStablePtr = unsafeCoerce . fakeStablePtr
\end{code}

The \lstinline!FakeStablePtr! type and the function by the same name are used
to mimic the underlying structure of \lstinline!StablePtr!. This makes its
exact implementation specific to the Haskell compiler in question, unlike
the ``proper'' solution based on actual stable pointers. The Haste
compiler, being based on GHC, has a very straightforward representation for
stable pointers, merely wrapping the ``machine'' level pointer in a single
layer of indirection, giving us the following implementation of fake stable
pointers:

\begin{code}
  data FakeStablePtr a = Fake !a
  
  fakeStablePtr = Fake
\end{code}

Thus, we may choose our implementation strategy depending on the capabilities
of our target compiler. For a single implementation targeting multiple
platforms however, proper stable pointers are the safer solution.

\subsection{Restricting imports to the IO monad}
The interface presented in this paper does not support importing pure
functions; any function originating in the host language must be safely locke
up within the IO monad. This may be seen as quite a drawback, as a host
language function operating solely over local state is definitely not
beyond the realms of possibility. Looking at our implementation of function
exports for pure functions, it seems that it would be possible to implement
imports in a similar way, and indeed we could.

However, ``could'' is not necessarily isomorphic to ``should''.
Foreign functions do, after all, come from the unregulated, disorderly world
outside the confines of the type checker. Haskell's type system does not
allow us to mix pure functions with possibly impure ones, and for good
reason. It is not clear that we should lift this
restriction just because a function is defined in another language.

Moreover, as explained in section \ref{sec:interface}, marshalling inbound
data is in many cases an inherently effectful operation, particularly when
involving complex data structures. Permitting the import of pure functions,
knowing fully well that a race condition exists in the time window between
the import's application and the resulting thunk's evaluation, does not strike
us as a shining example of safe API design.

\subsection{Continuation-based blocking in non-concurrent environments}
A particularly neat feature of the foreign function interface employed by the
GHCJS compiler is the ability for foreign host code to suspend execution while
waiting for an even to occur, even though its JavaScript host environment is
devoid of any concurrency support.\ \cite{ghcjs}
This is accomplished by giving imported functions an extra parameter:
a continuation to be called upon completion
of the foreign operation to resume execution of the Haskell program, instead
of simply returning like a ``normal'' JavaScript function would.

This functionality is not supported by our interface. GHCJS accomplishes this
by outputting continuation passing code which is executed by a clever
trampolining machinery. Supporting this feature would tie the interface to
a particular code generation strategy as well as add considerable complexity;
a price we deem too high to pay for this feature.

Instead, this functionality can be implemented on top of our interface without
much difficulty using a construct dubbed the ``poor man's concurrency monad''
\ \cite{poormansconcurrencymonad}; a monad implementing cooperative
multitasking with blocking synchronization variables in non-concurrent
environments.

\section{Related work}\label{sec:related}
Aside from the vanilla foreign function interface used as the basis of our
interface, there are several different, more modern, takes on interfacing
purely functional languages with host language code.
One common denominator is specialization: without exception, these
implementations rely in large part on modifications to the compiler or
language itself, in contrast to our interface which makes some sacrifices
in order to be implementable as a library over as wide a range of Haskell
dialects as possible.

\paragraph{Idris: host-parametric FFI}
\emph{Idris} is a dependently typed, Haskell-like language with backends for
several host environments, JavaScript being one of them.\ \cite{idris}
Like Haskell, Idris features monadic IO, but unlike Haskell, Idris' IO monad
\emph{is}, in a sense its foreign function interface.
IO computations are constructed from primitive building blocks, imported using
a function not unlike our \lstinline!host! function described in section
\ref{sec:interface}, and parameterized over the target environment.
This ensures that Idris code written specifically for a native environment
is not accidentally called from code targeting JavaScript and vice versa.

Idris' import function does not necessarily accept strings of foreign
language code, but is parameterized over the target environment just like the
IO monad; for JavaScript-targeting code, foreign code happens to be specified
as strings, but could conceivably consist of something more complex, such as
an embedded domain-specific language for building Idris-typed host language
functions.

\paragraph{Fay: featureful but static}
Our interface was partially inspired by the foreign function interface of the
\emph{Fay} language,
a ``proper subset of Haskell that compiles to JavaScript''.\ \cite{fay}
While the two are very similar in syntax, allowing users to import typed
strings of host language code, Fay's solution is highly specialized.
The compiler takes a heavy hand in the marshalling and import functionality,
parsing the host language code and performing certain substitutions on it.
While marshalling of arbitrary types is available, this marshalling is not
easily controllable by the user, but follows a sensible but fixed format
determined by the compiler.
This approach makes sense, as the interface is designed to support the Fay
language and compiler alone, but differs from our work which aims to create
a more generally applicable interface.

\paragraph{GHCJS: JavaScriptFFI}
The \emph{GHCJS} Haskell-to-JavaScript compiler\ \cite{ghcjs} utilizes the
relatively recent \emph{JavaScriptFFI} GHC extension,
which has unfortunately been rarely described outside a GHCJS context, to the
point of being conspicuously absent from even the GHC documentation.
Much like Fay, this extension parses and
performs substitutions over imported host language code to make imports
slightly more flexible, allowing for importing arbitrary expressions rather
than plain named functions.
It also enables additional safety levels for foreign
imports: \emph{safe}, where bad input data is replaced by default values and
foreign exceptions caught and marshalled into Haskell equivalents,
and \emph{interruptible}, which allows host language code to suspend execution
indefinitely even though JavaScript is completely single threaded.
This is accomplished by handing interruptible functions a continuation in
addition to their usual arguments, to call with the foreign function's
``return value'' as its argument when it is time for the foreign function
to return and let the Haskell program resume execution.

The JavaScriptFFI extension preserves the regular FFI's onerous restrictions
on marshallable types however, and while while GHCJS comes with convenience
functions to convert between these more complex types and the simple ones
allowed through the FFI, marshalling is not performed automatically and
functions in particular are cumbersome to push between Haskell and JavaScript.

\paragraph{UHC: the traditional FFI, on steroids for JavaScript}
The UHC Haskell compiler comes with a JavaScript backend as well, and matching
higher level extensions to its foreign function interface.\ \cite{uhc}
Like Fay, UHC provides automatic conversion of Haskell values to JavaScript
objects, as well as importing arbitrary JavaScript expressions, with some
parsing and wildcard expansion.
Also like Fay, the JavaScript representation produced by this
conversion is determined by the compiler, and not user configurable.
UHC does, however, provide several low level primitives for manipulating
JavaScript objects from within Haskell, both destructively and in a purely
functional manner.

\paragraph{Quasi-quotes}
Quasi-quotes represent another, more radically different, approach to the
problem of bridging with a host language.\ \cite{quasiquotes}
Allowing for the inline inclusion of large snippets of foreign code,
quasi-quotes have a lot in common with our interface, even eclipsing it in
power through anti-quotes, which allow the foreign code expressions to
incorporate Haskell data provided that the proper marshalling has been
implemented. Recent work by Manuel Chakravarty has extended the usefulness of
quasi-quotes even further, automating large parts of the stub generation
and marshalling required for using quasi-quoted host language code as a
foreign function interface.\ \cite{language-c-inline}

Unfortunately, this usefulness comes at the price of a rather complex
implementation. Not only is complex templating support from the compiler
required, but implementations need to be able to parse the host language as
well. A price in implementation effort that may not always be worth the extra
expressive power it buys.

\subsection{Future work}
By combining two optimizations given in section \ref{sec:optimizations},
the restriction of our \lstinline!host_safe! function to only accept statically
known strings and the elimination of calls to \lstinline!eval! for statically
known strings, it is possible to lift the restriction that our host language
support dynamic code evaluation: if all foreign imports are statically known,
and we are able to eliminate \lstinline!eval! calls for all statically known
functions, it follows that we are able to eliminate all \lstinline!eval!
calls.
While the actual implementation of this idea has yet to be worked out,
guaranteeing the complete absence of \lstinline!eval! from the generated host
code would remove the restriction that our host language supports dynamic code
evaluation at runtime, notably making our interface implementable on
recent versions of the Java Virtual Machine.
Implementing this interface for the Java Virtual Machine, with the
prerequisite Haskell-to-JVM compiler, would lend further weight to our
assertion that this interface is both both useful and broadly implementable.

Out interface does not currently catch and marshal host language
exceptions, but requires foreign language code to take care of any exceptions.
While the actual implementation is quite specific to a particular
host environment, automatically converting exceptions would be a useful feature
even so. Investigating the degree to which this feature could be implemented
in a host platform agnostic manner would be a possible extension of this work.

Due to the hard requirement that our host language be garbage collected,
our interface is not currently applicable in a C context. This is unfortunate,
as C-based host environments are still by far the most common for Haskell
programs. It may thus be worthwhile to investigate the compromises needed
to lift the garbage collection requirement from potential host environments.

\section{Conclusions}
\label{sec:conclusion}
We have presented the design of a novel, portable and platform agnostic foreign
function interface for hosted Haskell dialects. We have also given a number of
optimizations, improving the performance and safety of our interface
and lightening the restrictions placed on the host enfironment, and implemented
our interface as a library for the Haste Haskell-to-JavaScript compiler.
Finally, we have used this library to further extend our marshalling
capabilities to cover functions, as well as generic default marshalling for
arbitrary data types, contrasted out approach with a variety of existing
foreign function interfaces, and demonstrated that our library is relatively
performant.

While our interface is currently not applicable to Haskell implementations
targeting low level, C-like environments, it brings significant reductions in
boilerplate code and complexity for users needing to interface their Haskell
programs with their corresponding host environment in the space where it
\emph{is} applicable: Haskell implementations for high level target platforms.

The library is shipped in both source code and binary form as part of the
Haste distribution, available from the Haste Language website
at\ \cite{haste-lang.org}.

\acks
This work has been partially funded by the Swedish Foundation for Strategic Research, under grant RAWFP. Additional thanks to Koen Claessen, Emil Axelsson and Atze van der Ploeg for their valuable feedback and comments.

% We recommend abbrvnat bibliography style.

\bibliographystyle{abbrvnat}

% The bibliography should be embedded for final submission.

\begin{thebibliography}{}
\softraggedright

\bibitem[Augustsson, Massey (2013)]{printf}
  Augustsson, Lennart, and Massey, Bart. The {\tt Text.Printf} module.
  Available from \url{http://hackage.haskell.org/package/base-4.8.0.0/docs/Text-Printf.html},
  2015-04-01.
\bibitem[Balat, Chambart, Henry (2012)]{ocsigen}
  Balat, Vincent, Chambart, Pierre, and Henry, Grégoire.
  "Client-server Web applications with Ocsigen." In WWW2012, p. 59. 2012.
\bibitem[Brady (2015)]{idris}
  Brady, Edwin.
  "Cross-platform Compilers for Functional Languages."
  Under consideration for Trends in Functional Programming, 2015.
\bibitem[Chakravarty et al (2003)]{ffi}
  Chakravarty, M. M. (Ed.). (2003).
  The Haskell Foreign Function Interface 1.0: An Addendum to the Haskell 98
  Report.
\bibitem[Chakravarty (2014)]{language-c-inline}
  Chakravarty, M. M. (2014, September).
  Foreign inline code: systems demonstration.
  In Proceedings of the 2014 ACM SIGPLAN symposium on Haskell (pp. 119-120).
  ACM.
\bibitem[Chakravarty, Keller, Peyton Jones (2005)]{typefamilies}
  Chakravarty, M. M., Keller, G., \& Peyton Jones, S. (2005, September).
  Associated type synonyms.
  In ACM SIGPLAN Notices (Vol. 40, No. 9, pp. 241-253). ACM.
\bibitem[Claessen (1999)]{poormansconcurrencymonad}
  Claessen, K. (1999).
  A poor man's concurrency monad.
  Journal of Functional Programming, 9(03), 313-323.
\bibitem[Dijkstra, Stutterheim, Vermeulen, Swierstra (2013)]{uhc}
  Dijkstra, A., Stutterheim, J., Vermeulen, A., \& Swierstra, S. D. (2013).
  Building JavaScript applications with Haskell.
  In Implementation and Application of Functional Languages (pp. 37-52).
  Springer Berlin Heidelberg.
\bibitem[Done (2015)]{fay}
  Done, C.
  Fay programming language.
  Available from \url{http://fay-lang.org/}, 2015-05-15.
\bibitem[The 2015 ECMAScript Language Specification, Rev. 38 (2015)]{es6}
  The 2015 ECMAScript Language Specification, Rev. 38, Final Draft.
  Available from \url{http://wiki.ecmascript.org/lib/exe/fetch.php?id=harmony%3Aspecification_drafts&cache=cache&media=harmony:ecma-262_6th_edition_final_draft_-04-14-15.pdf},
  (2015-05-13).
\bibitem[Eisenberg et al (2014)]{closedtypefamilies}
  Eisenberg, R. A., Vytiniotis, D., Peyton Jones, S., \& Weirich, S. (2014).
  Closed type families with overlapping equations.
  ACM SIGPLAN Notices, 49(1), 671-683.
\bibitem[Ekblad, Claessen (2014)]{haste}
  Ekblad, A., \& Claessen, K. (2014, September).
  A seamless, client-centric programming model for type safe web applications.
  In Proceedings of the 2014 ACM SIGPLAN symposium on Haskell (pp. 79-89). ACM.
\bibitem[Epstein, Black, Peyton Jones (2011)]{cloudhaskell}
  Epstein, J., Black, A. P., \& Peyton Jones, S. (2011, September).
  Towards Haskell in the cloud.
  In ACM SIGPLAN Notices (Vol. 46, No. 12, pp. 118-129). ACM.
\bibitem[GHC User's Guide (2015)]{ghc710}
  The GHC Team.
  "The Glorious Glasgow Haskell Compilation System User’s Guide, Version 7.10.1",
  pp. 256-257.
  Available from \url{http://www.haskell.org/ghc/docs/7.10.1/users_guide.pdf},
  2015-04-01.
\bibitem[The Haste language website (2015)]{haste-lang.org}
  The Haste language website.
  Available from \url{http://haste-lang.org}, 2015-05-14.
\bibitem[Mainland (2007)]{quasiquotes}
  Mainland, G. (2007, September).
  Why it's nice to be quoted: quasiquoting for haskell.
  In Proceedings of the ACM SIGPLAN workshop on Haskell workshop (pp. 73-82).
  ACM.
\bibitem[Magalhães, Dijkstra, Jeuring, Löh (2010)]{generics}
  Magalhães, J. P., Dijkstra, A., Jeuring, J., \& Löh, A. (2010).
  A generic deriving mechanism for Haskell.
  In Haskell 2010, Baltimore, MD, USA, 30 September 2010, pages 37–48. ACM, 2010.
\bibitem[Nazarov, Mackenzie, Stegeman (2015)]{ghcjs}
  Nazarov, V., Mackenzie, H., \& Stegeman, L.
  "GHCJS Haskell to JavaScript Compiler".
  Available from \url{https://github.com/ghcjs/ghcjs}, (2014-05-14).
\bibitem[Peyton Jones, Tolmach, Hoare (2001)]{rewriterules}
  Jones Peyton, S., Tolmach, A., \& Hoare, T. (2001, September).
  Playing by the rules: rewriting as a practical optimisation technique in GHC.
  In Haskell workshop (Vol. 1, pp. 203-233).
\bibitem[Reid (1994)]{stableptr}
  Reid, A. (1994, September).
  Malloc pointers and stable pointers: Improving Haskell's foreign language
  interface.
  In Glasgow Functional Programming Workshop Draft Proceedings, Ayr, Scotland.
\bibitem[Yakeley, 2014]{time}
  Yakeley, A.
  The \emph{time} package.
  Available from \url{from http://hackage.haskell.org/package/time}, 2015-04-01.

\end{thebibliography}


\end{document}
